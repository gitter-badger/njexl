\chapter{Preface}
 
{\LARGE n}Jexl initially was a continuation of \href{http://commons.apache.org/proper/commons-jexl}{Apache Jexl} project.
That project was not active for 17 months, and I needed to use it, so I forked it. The reason are to be found below.

\begin{subsection}{A brief History}

All I wanted is a language where I can write my test automation freely - i.e. using theories from testing.
The standard book, and there is only one for formal software testing is 
that of \href{http://www.amazon.com/Software-Testing-Techniques-2nd-Edition/dp/1850328803}{Beizer}.

There was no language available which lets me intermingle with Java POJOs and let me write my test automation (validation and verifications). 
Worse still - one can not write test automation freely using Java. 
As almost all of modern enterprise application are written using Java, it is impossible to avoid Java 
and write test automation : in many cases you would need to call appropriate Java methods to automate APIs.

Thus, one really needs a JVM scripting language that can freely call and act on POJOs.
The idea of extending JEXL thus came into my mind : a language that has all the good stuffs from
the vast Java libraries, but clearly not verbose enough.

After cloning JEXL - and modifying it real heavy - a public release in a public repository
seemed a better approach. There, multiple people can look into it, rather than one lone ranger working from his den.
And hence nJexl was born. The \emph{n} stands for \emph{Neo,New}, not \emph{Noga}, which is, by the way, my nick name.

\end{subsection}

\begin{subsection}{An Open Challenge From Java Land}

This actually was an \href{http://steve-yegge.blogspot.in/2006/03/execution-in-kingdom-of-nouns.html}{open challenge}:


\emph{A noted PhD from Sun, read this essay, and had this to say: 
``hmm, chuckle :) This guy has too much time on his hands ! he should be doing useful work, or inventing a new language to solve the problems. Its easy to throw stones - harder to actually roll up your sleeves and fix an issue or two, or write/create a whole new language, and then he should be prepared to take the same criticism from his peers the way he's dishing it out for others. Shame - I thought developers were constructive guys and girls looking to make the lives of future software guys and girls easier and more productive, not self enamouring pseudo-intellectual debaters, as an old manager of mine used to say in banking IT - 'do some work' !''}


And I like to think that I just did. That too, while at home, in vacation time, and in night time ( from 8 PM to 3 AM. - see the check-ins.)
Alone. But the users are the best judge.

\end{subsection}


\begin{subsection}{About The Language}

It is an interpreted language. It is asymptotically as fast as python, with a general lag of 200 ms of reading and parsing files, where native python is faster. After that the speed is the same.
 
It is a multi-paradigm language. It supports functionals ( i.e. anonymous functions ) out of the box, and every function by design can take functional as input. There are tons of in built methods which uses functional.

It supports OOP. Albeit not recommended, as \href{http://harmful.cat-v.org/software/OO_programming/why_oo_sucks}{OOPs!} tells you.
clearly shows why. In case you want C++ i.e.  \href{http://en.wikipedia.org/wiki/Multiple_inheritance}{multiple inheritance} with full operator overloading, friend functions, etc. then this is for you anyways. 
Probably you would love it to the core.

Python is a brilliant language, and I shamelessly copied many, and many adages of Python here. The heavy use of  \_\_xxx\_\_    
literals, and the \emph{me} directive, and \emph{def} is out and out python. 

The space and tab debate is very religious, and hence JEXL is "{ blocked }" : Brace yourself.
Pick tab/space to indent - none bothers here.
You can use ";" to separate statements in a line. 
Lines are statements.

\end{subsection}



And finally, all of these pages were typed using my trust worthy Mac Book Pro, at home using \href{http://pages.uoregon.edu/koch/texshop}{TexShop-64}.
So, thanks to Richard Koch, Max Horn Dirk Olmes. For \href{http://tug.org/mactex/}{MacTex}, thanks MacTex. You guys are great!
Thanks to Apple for creating such a beautiful systems to work on. Steve, I love you. RIP.
For Windows, the trusty \href{http://www.tug.org/texworks}{TexWorks} was used using \href{http://miktex.org}{MikeTex} packages. Thanks to you too!
In the end many a thanks to Gabriel Hjort Blindell - for the beautiful style file he created which can be found \href{http://web.ict.kth.se/~ghb/publications/documents/thesis-template.tar.gz}{here}. Gabriel, thanks a ton.
