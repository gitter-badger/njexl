\chapter{Comprehensions on Collections}\label{comprehension-collection}

{\LARGE C}omprehension is a method which lets one create newer collection from older ones.
In this chapter we would see how different collections can be made from existing collections.
\index{Comprehensions}

\begin{section}{Using Anonymous Argument}

\begin{subsection}{Arrays and List}
The general form for list and arrays can be written ( as from chapter 2 ):
\begin{lstlisting}[style=JexlStyle]
def map(item) {/* does some magic */}
def comprehension(function, items){
   c = collection()
   for ( item : items ){
     c_m = function(c)
     c.add( c_m ) 
   }
   return c
}
//or this way :
c = collection{ function($) }(items)
\end{lstlisting}
Obviously $list()$ generates $list$ and $array()$ generates $array$.
\index{Comprehensions : array(), list() }
Hence, these are valid :
\begin{lstlisting}[style=JexlStyle]
an = array{ $ + 2 }([1,2,3,4]) // [ 3,4,5,6 ]
ln = list{ $ ** 2 }([1,2,3,4]) // [1,4,9,16 ]
\end{lstlisting}

So, the result of the anonymous block is taken as a function to map the item into a newer item, 
and finally added to the final collection.

\end{subsection}

\begin{subsection}{Set}
\index{Comprehensions : set()}

Set also follows the same behavioural pattern just like $array()$ and $list()$, but one needs 
to remember that the $set$ collection does not allow duplicates. Thus:

\begin{lstlisting}[style=JexlStyle]
s = set{ $ % 3 }( [1,2,3,4,5] ) // [0,1,2]
\end{lstlisting}

Basically, the $map$ function for the $set$ construct, defines the key for the item.
Thus, unlike its cousins $list()$ and $array()$, the $set()$ function may return a collection
with size less than the input collection. 

\end{subsection}

\begin{subsection}{Dict}
\index{Comprehensions : dict()}

Dictionaries are special kind of a collection with $(key,value)$ pair with unique keys.
Thus, creating a dictionary would mean uniquely specifying the key, value pair.
That can be specified either in a dictionary tuple way, or simply a pair way: 

\begin{lstlisting}[style=JexlStyle]
// range can be passed in for collection
d = dict{ t = { $ : $**2 }  }([0:4])
/* d = {0 : 0, 1 : 1, 2 : 4, 3:9 }  */
d = dict{ [ $ , $**2 ]  }([0:4]) // same as previous
\end{lstlisting}

There is another way of creating a dictionary, that is, 
passing two collections of same size, and making first collection 
as keys, mapping it to the corresponding second collection as values:

\begin{lstlisting}[style=JexlStyle]
k = ['a','b','c' ]
v = [1,2,3]
d = dict(k,v)
/*  d = { 'a' : 1 , 'b' : 2, 'c' : 3 }  */ 
\end{lstlisting}

This is formally the operation called \href{https://en.wikipedia.org/wiki/Pairing}{pairing}:
$$
D : (K,V) \to T \; ; \; t_i = ( k_i, v_i ) \; ; \; k_i \in K \; ; \; v_i \in V 
$$

\end{subsection}

\end{section}


\begin{section}{Constructs Altering the Iteration Flow}
In normal iterative languages, there are \emph{continue} and \emph{break}.
The ideas are borrowed from them, and improved to give a declarative feel onto it.

\begin{subsection}{Continue and Break}
\index{Flow Control : continue, break }

The idea of continue would be as follows :
\begin{lstlisting}[style=JexlStyle]
for ( i : items ){
    if ( condition(i) ){
      // execute some 
      continue 
    }
    // do something else 
}
\end{lstlisting}
That is, when the condition is true, execute the code block, 
and then continue without going down further ( not going to do something else ). 
The idea of the break is :
\begin{center}\begin{minipage}{\linewidth}
\begin{lstlisting}[style=JexlStyle]
for ( i : items ){
    if ( condition(i) ){ 
    /* execute some code */  
    break 
    }
    // do something else here
}
\end{lstlisting}
\end{minipage}\end{center}

That is, when the condition is true, execute the code block, 
and then break without proceeding with the loop further. 
Evidently they change the flow of control of the iterations.
\end{subsection}

\begin{subsection}{Continue}
\index{Flow Control : continue}
As we can see, the $condition()$ is implicit in both $break$ and $continue$,
in nJexl this has become explicit. Observe that:  

\begin{lstlisting}[style=JexlStyle]
for ( i : items ){
    continue( condition(i) ){ /* continue after executing this */} 
    // do something else 
}
\end{lstlisting}
is equivalent of what was being shown in the previous subsection.
As a practical example, lets have some fun with the problem of $FizzBuzz$,
that is, given a list of integers, if $n$ is divisible by $3$ print $Fizz$,
if something is divisible by $5$, print $Buzz$, and for anything else print the number $n$. 
A solution is :

\begin{lstlisting}[style=JexlStyle]
for ( n : integers ){
    continue( n % 15 == 0 ){ write('FizzBuzz') }
    continue( n % 3 == 0 ){ write('Fizz') }
    continue( n % 5 == 0 ){ write('Buzz') }
    write(n) 
}
\end{lstlisting}
Obviously, the block of the continue is optional.
Thus, one can freely code the way it was mentioned in the 
earlier subsection.
\end{subsection}


\begin{subsection}{Break}
\index{Flow Control : break}
As mentioned in the previous subsection, 
$break$ also has the same features as continue. 

\begin{lstlisting}[style=JexlStyle]
for ( i : items ){
    break( condition(i) ){ /* break loop after executing this */} 
    // do something else 
}
\end{lstlisting}
is equivalent of what was being shown in the previous subsection.
As a practical example, lets find if two elements of two lists
when added together generates a given value or not.
Formally :

$$
\exists (a,b) \in A \times B \; ; \; s.t. \; a + b = c  
$$

\begin{lstlisting}[style=JexlStyle]
A = [ 1, 4, 10, 3, 8 ]
B = [ 2, 11, 6 , 9 ]
c = 10 
for ( p : A*B  ){
    break( p[0] + p[1] ==  c ){ 
    write( '%d %d\n' , p[0] , p[1] ) } 
}
\end{lstlisting}
Obviously, the block of the break is optional.
\end{subsection}

\begin{subsection}{Substitute for Select}
\index{Flow Control : usage of continue}
One can readily use the $break$ and $continue$ in larger scheme of comprehensions.
Observe this :

\begin{lstlisting}[style=JexlStyle]
l_e1 = select{ $ % 2 == 0 }([0:10])
l_e2 = list{ continue($ % 2 != 0) ; $ }([0:10])
\end{lstlisting}
Both are the same list. Thus, a theoretical conversion rule is :

\begin{lstlisting}[style=JexlStyle]
ls = select{ <condition> }(collection)
lc = list{ continue( not ( <condition> ) ) ; $ }(collection)
\end{lstlisting}
now, the $ls == lc$, by definition.
More precisely, for a generic $select$ with $where$ clause:

\begin{lstlisting}[style=JexlStyle]
ls = select{ where ( <condition> ){ <body> } }(collection)
lc = list{ continue( not (<condition>) ) ; <body> }(collection)
\end{lstlisting}
Obviously, the $list$ function can be replaced with any $collection$ type : $array, set, dict$.

Similarly, $break$ can be used to simplify conditions :

\begin{lstlisting}[style=JexlStyle]
l_e1 = select{ $ % 2 == 0 and $ < 5 }([0:10])
l_e2 = list{ break( $ > 5 ) ; 
       continue($ % 2 != 0) ; $ }([0:10])
\end{lstlisting}

Note that Break body is inclusive:
  
\begin{lstlisting}[style=JexlStyle]
l = list{ break( $ > 3 ) { $ } ; $ }([0:10])
/* l = [ 1,2,3, 4] */
\end{lstlisting}
 
So, with a body, the boundary value is included into $break$.
\end{subsection}

\begin{subsection}{Uses of Partial}
One can use the \emph{PARTIAL} for using one one function to substitute another,
albeit loosing efficiency.
Below, we use list to create \emph{effectively} a set:

\begin{lstlisting}[style=JexlStyle]
l = list{ continue( $ @ _$_ ) ; $ }( [1,1,2,3,1,3,4] )
/* l = [ 1,2,3, 4] */
\end{lstlisting}

But it gets used properly in more interesting of cases.
What about finding the prime numbers using the \href{https://en.wikipedia.org/wiki/Sieve_of_Eratosthenes}{Sieve of Eratosthenes}?
We all know the drill imperatively:
\index{Sieve of Eratosthenes : imperativel}

\begin{lstlisting}[style=JexlStyle]
def is_prime( n ){
    primes = set(2,3,5)
    for ( x : [6:n+1] ){
        x_is_prime = true 
        for ( p : primes ) {
            break( x % p == 0  ){ x_is_prime = false }
        }
        if ( x_is_prime ){
          if ( x == n ) return true 
          primes += x 
        }
    }
    return false 
}
\end{lstlisting}
Now, a declarative style coding :
\index{Sieve of Eratosthenes : declarative}
\begin{lstlisting}[style=JexlStyle]
def is_prime( n ){
    primes = set{
    // store the current number
        me = $
        // check : *me* is not prime - using partial set of primes
        not_prime_me = ( index{  me % $ == 0 }( _$_ ) >= 0 ) 
        // if not a prime, continue 
        continue( not_prime_me )
        // collect me, if I am prime 
        $   
    }( [2:n+1] )
    // simply check if n belongs to primes
    return ( n @ primes )  
}
\end{lstlisting}
Observe that, if we remove the comments, it is a one liner. 
It really is. Hence, declarative style is indeed succinct, and very powerful.

\end{subsection}

\end{section}



\begin{section}{Comprehensions using Multiple Collections : Join}
\index{Multiple Collection Comprehension : join()}

In the previous sections we talked about the comprehensions using a single collection.
But we know that there is this most general purpose comprehension, that is, 
using multiple collections. This section we introduce the $join()$ function.

\begin{subsection}{Formalism}
Suppose we have multiple sets $S_1,S_2,...S_n$.
Now, we want to generate this collection of tuples 
$$
e = <e_1,e_2,...,e_n> \; \in S_1 \times S_2 \times ... \times S_n
$$ 
such that the condition ( \href{https://en.wikipedia.org/wiki/Predicate\_(mathematical\_logic)}{predicate} ) $P(e) = true $. \index{predicate}
This is what join really means. Formally, then :
$$
J(P,e) := \{  e \in S_1 \times S_2 \times ... \times S_n \; | \; P(e) \} 
$$

\end{subsection}

\begin{subsection}{As Nested Loops}
Observe this, to generate a cross product of collection $A,B$, we must
go for nested loops like this:

\begin{lstlisting}[style=JexlStyle]
for ( a : A ){
   for ( b : B ) {
     write ( '(%s,%s)', a,b ) 
   }
}
\end{lstlisting}
This is generalised for a cross product of $A,B,C$, and generally into any cross product. 
Hence, the idea behind generating a tuple is nested loop.

Thus, the $join()$ operation with $condition()$ predicate essentially is :

\begin{lstlisting}[style=JexlStyle]
collect = collection()
for ( a : A ){
   for ( b : B ) {
      tuple = [a,b]
      if ( condition ( tuple ) ) { collect += tuple  } 
   }
}
\end{lstlisting}
\end{subsection}

\begin{subsection}{Join Function}
\index{join()}
Observe that the free join, that is a join without any predicate, 
or rather join with predicate defaults to true is really a full cross product, 
and is available using the power operator :

\begin{lstlisting}[style=JexlStyle]
A = ['a','b']
B = [ 0, 1 ]
j1 = A * B // [ [a,0] , [a,1] ,[b,0], [b,1] ]
j2 = join( A , B ) // [ [a,0] , [a,1] ,[b,0], [b,1] ]
\end{lstlisting}

But the power of $join()$ comes from the predicate expression one can pass in 
the anonymous block. Observe now, if we need 2 permutations of a list of 3 : $^3P_2$ :

\begin{lstlisting}[style=JexlStyle]
A = ['a','b', 'c' ]
// generate 2 permutations 
p2 = join{  $.0 != $.1 } ( A , A ) 
\end{lstlisting}
 
\end{subsection}

\begin{subsection}{Finding Permutations }
\index{Permutations}

In the last subsection we figured out how to find 2 permutation.
The problem is when we move beyond 2, the condition can not be aptly specified as $\$.0 \ne \$.1 $.
It has to move beyond. So, for $^3P_3$ we have :

\begin{lstlisting}[style=JexlStyle]
A = ['a','b', 'c' ]
// generate 3 permutations 
p2 = join{ #|set($)| == #|$|  } ( A , A,  A ) 
\end{lstlisting}

which is the declarative form for permutation, given all elements are unique. 
\end{subsection}

\begin{subsection}{Searching for a Tuple }
\index{Searching for a tuple}

In the last section we were searching for a tuple $t$ from two collections $A,B$,
such that $c = t.0 + t.1$. We did that using power and break, now 
we can do slightly better:

\begin{lstlisting}[style=JexlStyle]
A = [ 1, 4, 10, 3, 8 ] 
B = [ 2, 11, 6 , 9 ]
c = 10
v = join{ break( $.0 + $.1 == c) }(A,B) 
/* v := [[1,9]] */
\end{lstlisting}

which is the declarative form of the problem. 
\end{subsection}

\begin{subsection}{Finding Combinations}
\index{Combinations}

We did find permutation. What about combinations?
What about we want to find combinations from 3 elements, 
taken 2 at a time? We observe that every combination is still a permutation.
So, once it qualified as a permutation, we need to check that if the pair
is in strictly increasing order or not (sorted). Thus :

\begin{lstlisting}[style=JexlStyle]
A = [ 'a' ,'b', 'c' ] 
c = join{ // check permutation , then 
       #|set($)| == #|$| and 
       // check sorted - same trick we used earlier too!      
       index{ _ > 0 and  $$[_-1] > $ }($) < 0 
    }(A,A)
\end{lstlisting}
This is how we can solve the problem with the aid of a declarative form. 
\end{subsection}

\begin{subsection}{Projection on Collections}
\index{project}

Sometimes it is needed to create a sub-collection from the collection. 
This, is known as projection ( choosing specific columns of a row vector ).
The idea is simple :

\begin{lstlisting}[style=JexlStyle]
a = [0,1,2,3,4]
sub(a,2) // [2, 3, 4]
sub(a,2,3) // [2, 3]
sub(a,-1) // [0, 1, 2, 3]
sub(a,0,-1) // [0, 1, 2, 3]
sub(a,1,-1) // [1, 2, 3]
\end{lstlisting}
If you do not like the name $sub()$ , the same function is aliased under $project$.

While project works on a range \emph{(from,to)}, the generic idea can be expanded. 
What if I want to create a newer collection from a collection using a $range$ type? 
Clearly I can : \index{collection : slicing }

\begin{lstlisting}[style=JexlStyle]
a =[0, 1, 2, 3, 4]
a[[0:3:2]] // @[0, 2]
\end{lstlisting}

Now, given a set of indices, one can define a generic project operation, 
but that is very easy to do using collection comprehensions :

\begin{lstlisting}[style=JexlStyle]
a = [0, 1, 2, 3, 4]
indices = [0,2]
p_a = array{ continue( _ != $ ) ; a[$] }(indices)  
\end{lstlisting}
This simply selects those indices in order from the collection $a$.
In fact, defining \emph{project()} or \emph{sub()} is easy :

\begin{lstlisting}[style=JexlStyle]
a = [0, 1, 2, 3, 4]
indices = [from,to]
p_a = array{ continue( _ < indices.0 or _ > indices.1 ) ; $ }(a)  
\end{lstlisting}

\end{subsection}

\end{section}

