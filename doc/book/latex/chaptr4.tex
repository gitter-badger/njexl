\chapter{Formal Constructs}\label{using-predicates}

{\LARGE F}ormalism is the last thing that stays in a software developers mind nowadays. This is a result of mismanaging expectation - software may be art, but is not science at all. Most of the cases, a bit of formal thinking solves a lot of the problems that can come.
This section would be specifically to understand what sort of formalism we can use in practice.

\begin{section}{Conditional Building Blocks}

\begin{subsection}{Formalism}
\index{Predicate Logic}
In a formal logic like FOPL, the statements can be made from the basic ingredients, 
\emph{``there exist''} and \emph{``for all''}. We need to of course put some logical stuff like $AND$, $OR$,
and some arithmetic here and there, but that is what it is.

If we have noted down the \emph{sorting problem} in chapter 2, we would have restated the problem as :

\begin{center}
\emph{ There does not exist an element which is less than that of the previous one. }
\end{center} 

In mathematical logic, \emph{``there exists''} becomes : $\exists$ and \emph{``for all''} becomes $\forall$.
and logical not is shown as $\neg$ ,  So, the same formulation goes in : let 
$$
S = \{ 0, 1, 2, ... , size(a)-1  \} =  \{  x \in \mathbb{N} | x < size(a) \} 
$$
then :
$$
  i \in S ; \forall \; i  \neg \; ( \exists \; a[i] \; s.t. \; a[i] < a[i-1]    ) 
$$ 

And the precise formulation of what is a sorted array/list is done.

\end{subsection}


\begin{subsection}{There Exist In Containers}
\index{Predicate Logic : there exists}
To check, if some element is, in some sense exists inside a container ( list, set, dict, array, heap )
one needs to use the $IN$ operator, which is $@$. \index{operator : \@}

\begin{lstlisting}[style=JexlStyle]
l = [1,2,3,4] // l is an array 
in = 1 @ l // in is true 
in = 10 @ l // in is false 
d = { 'a' : 10 , 'b' : 20  }
in = 'a' @ d // in is true 
in = 'c' @ d // in is false 
in = 10 @ d // in is false 
in = 10 @ d.values() // in is true 
s = "hello"
in = "o" @ s // in is true 
/* This works for linear collections even */
m = [2,3]
in = m @ l // true 
in = l @ l // true  
\end{lstlisting}

This was not simply put in the place simply because we dislike $x.contains(y)$, in fact we do.
We dislike the form of object orientation where there is no guarantee that $x$ would be null or not.
Formally, then :

\begin{lstlisting}[style=JexlStyle]
// equivalent function 
def in_function(x,y){
   if ( empty(x) ) return false 
   return x.contains(y) 
}
// or one can use simply 
y @ x // same result
\end{lstlisting}
\end{subsection}

\begin{subsection}{Size of Containers : empty, size, cardinality}
To check whether a container is empty or not, 
use the $empty()$ \index{empty()} function, just mentioned above. Hence:
\begin{lstlisting}[style=JexlStyle]
n = null
e = empty(n) // true 
n = []
e = empty(n) // true 
n = list()
e = empty(n) // true 
d = { : }
e = empty(d) // true 
nn = [ null ]
e = empty(nn) // false 
\end{lstlisting}

For the actual size of it, there are two alternatives.
One is the $size()$ \index{size()}  function :

\begin{lstlisting}[style=JexlStyle]
n = null
e = size(n) // -1 
n = []
e = size(n) // 0 
n = list()
e = size(n) // 0 
d = { : }
e = size(d) // 0 
nn = [ null ]
e = size(nn) // 1 
\end{lstlisting}

Observe that it returns negative given input null. 
That is a very nice way of checking null.
The other one is the $cardinal$ operator \index{operator : cardinality }:

\begin{lstlisting}[style=JexlStyle]
n = null
e = #|n| // 0 
n = []
e = #|n| // 0 
n = list()
e = #|n| // 0 
d = { : }
e = #|d| // 0 
nn = [ null ]
e = #|nn| // 1 
\end{lstlisting}
This operator in some sense gives a measure. It can not be negative, 
so cardinality of $null$ is also 0. 
\end{subsection}

\begin{subsection}{There Exist element with Condition : index, rindex }
Given we have if \emph{``y in container x''} or not,
what if when asked a question like : \emph{ ``is there a y in x such that f(y) is true''} ?
This pose a delicate problem. 

Notice this is the same problem we asked about sorting. Is there an element (y) in x, 
such that the sorting order is violated? If not, the collection is sorted.
This brings back the $index$ function \index{index()}.
We are already familiar with the usage of $index()$ function from chapter 2.
But we would showcase some usage :
\index{Predicate Logic : there exist : index() }
 
\begin{lstlisting}[style=JexlStyle]
l = [ 1, 2, 3, 4, 5, 6 ]
// search an element such that double of it is 6
i = index{ $ * 2 == 6 }(l) // i : 2 
// search an element such that it is between 3 and 5 
i = index{ $ < 5 and $ > 3 }(l) // i : 3 
// search an element such that it is greater than 42
i = index{  $ > 42 }(l) // i : -1, failed 
\end{lstlisting}

The way index function operates is: the statements inside the anonymous block are executed.
If the result of the execution is true, the index function returns the index in the collection.
If none of the elements matches the true value, it returns $-1$.
Index function runs from left to right, and there is a variation $rindex()$ \index{index()}
which runs from right to left.

\index{Predicate Logic : there exist : rindex() }

\begin{lstlisting}[style=JexlStyle]
l = [ 1, 2, 3, 4, 5, 6 ]
// search an element such that it is greater than 3  
i = index{ $ > 3 }(l) // i : 3
// search an element such that it is greater than 3  
i = rindex{ $ > 3 }(l) // i : 5 
// search an element such that it is greater than 42
i = rindex{  $ > 42 }(l) // i : -1, failed 
\end{lstlisting}

Thus, the \emph{there exists} formalism is taken care by these operators and functions together.

\end{subsection}


\begin{subsection}{For All elements with Condition : select }
\index{Predicate Logic : for all : select() }

We need to solve the problem of \emph{for all}. This is done by $select()$ function \index{select()}.
The way select function works is : executes the anonymous statement block, and if the condition is true, 
then select and collect that particular element, and returns a list of collected elements.

\begin{lstlisting}[style=JexlStyle]
l = [ 1, 2, 3, 4, 5, 6 ]
// select all even elements 
evens = select{ $ % 2 == 0 }(l)
// select all odd elements 
odds = select{ $ % 2 == 1 }(l) 
\end{lstlisting}

\end{subsection}

\begin{subsection}{Partition a collection on Condition : partition }
Given a $select()$, we are effectively partitioning the collection into two halves,
$select()$ selects the matching partition. In case we want both the partitions, then 
we can use the $partition()$ function \index{partition()}.

\begin{lstlisting}[style=JexlStyle]
l = [ 1, 2, 3, 4, 5, 6 ]
#(evens,odds) = partition{ $ % 2 == 0 }(l)
write(evens) // prints 2, 4, 6 
write(odds) // prints 1, 3, 5
\end{lstlisting}
\end{subsection}

\begin{subsection}{Collecting value on Condition : where }
Given a $select()$ or $partition()$, we are collecting the values already 
in the collection. What about we want to change the values? 
This is done by the conditional known as $where()$ \index{where()}.
The way where works is, we put the condition inside the where.
The result would be the result of the condition, but the where clause body 
would be get executed. Thus, we can change the value we want to collect
by replacing the $ITEM$ variable, that is $\$$.

\begin{lstlisting}[style=JexlStyle]
l = [ 1, 2, 3, 4, 5, 6 ]
#(evens,odds) = partition{ where($ % 2 == 0){ $ = $**2  }  }(l)
write(evens) // prints 4, 16, 36 
write(odds) // prints 1, 3, 5
\end{lstlisting}

This is also called \emph{item rewriting} \index{item rewriting}.
\end{subsection}

\end{section}


\begin{section}{Operators from Set Algebra}

\begin{subsection}{Set Algebra}
\index{Set Algebra}

Set algebra, in essence runs with the notions of the following ideas :
\begin{enumerate}
\item{There is an unique empty set.}
\item { The following operations are defined :
\begin{enumerate}{
     \item{Set Union is defined as : \index{Set Algebra : Union}
$$
U_{AB} = A \cup B \; :=  \{ x \in A \; or \; x \in B \} := U_{BA}
$$ 
}
    \item{Set Intersection is defined as : \index{Set Algebra : Intersection}
$$
I_{AB} = A \cap B \; :=  \{ x \in A \; and \; x \in B \} := I_{BA}
$$
}
\item{Set Minus is defined as : \index{Set Algebra : Minus}
$$
M_{AB} := A \setminus B \; :=  \{ x \in A \; and \; x \not \in B \} 
$$
}
\item{Set Symmetric Difference is defined as : \index{Set Algebra : Symmetric Difference} 
$$
\Delta_{AB} := ( A \setminus B ) \cup ( B  \setminus A ) := \Delta_{BA}
$$
}
\item{Set Cross Product is defined as : 
$$
X_{AB} = \{ ( a, b ) \; | \; a \in A \; and \;  b \in B  \} 
$$
}
}
\end{enumerate}}
\item{ The following relations are defined:
\begin{enumerate}

\item{ Subset Equals : \index{Set Algebra : Subset Equals}
$$
A \subseteq B \; when \;  x \in A \implies x \in B
$$
}

\item{ Equals : \index{Set Algebra : Equals}
$$
A = B \; when \; A \subseteq B \; and \; B \subseteq A
$$
}


\item{ Proper Subset : \index{Set Algebra : Subset,Proper}
$$
A \subset B \; when \; A \subseteq B \; and \; \exists x \in B \;s.t.\; x \not \in A
$$
}

\item{ Superset Equals: \index{Set Algebra : Superset Equals}
$$
A \supseteq B \; when \; B \subseteq A 
$$
}

\item{ Superset : \index{Set Algebra : Superset,Proper}
$$
A \supset B \; when \; B \subset A 
$$
}
\end{enumerate}}
\end{enumerate}
\end{subsection}

\begin{subsection}{Set Operations on Collections}
\index{Set Operations : Collection}

For the sets, the operations are trivial.
\begin{lstlisting}[style=JexlStyle]
s1 = set(1,2,3,4)
s2 = set(3,4,5,6)
u = s1 | s2 // union is or : u = { 1,2,3,4,5,6}
i = s1 & s2 // intersection is and : i = { 3,4 }
m12 = s1 - s2 // m12 ={1,2}
m21 = s2 - s1 // m12 ={5,6}
delta = s1 ^ s2 // delta = { 1,2,5,6}
\end{lstlisting}

For the lists or arrays, where there can be multiple elements present, 
this means a newer formal operation.
Suppose in both the lists, an element `e' is present, in $n$ and $m$ times.
So, when we calculate the following :
\begin{enumerate}
\item{Intersection :
  the count of $e$ would be $min(n,m)$.
}
\item{Union :
  the count of $e$ would be $max(n,m)$.
}
\item{Minus :
  the count of $e$ would be $max(0,n-m)$.
}

\end{enumerate}
With this, we go on for lists:

\begin{lstlisting}[style=JexlStyle]
l1 = list(1,2,3,3,4,4)
l2 = list(3,4,5,6)
u = l1 | l2 // union is or : u = { 1,2,3,3,4,4,5,6}
i = l1 & l2 // intersection is and : i = { 3,4 }
m12 = l1 - l2 // m12 ={1,2,3,4}
m21 = l2 - l1 // m12 ={5,6}
delta = l1 ^ l2 // delta = { 1,2,3,4,5,6}
\end{lstlisting}

Now, for dictionaries, the definition is same as lists, 
because there the dictionary can be treated as a list of key-value pairs.
So, for one pair to be equal to another, both the key and the value must match.
Thus:

\begin{lstlisting}[style=JexlStyle]
d1 = {'a' : 10, 'b' : 20 , 'c' : 30 }
d2 = {'c' : 20 , 'd' : 40  }
// union is or : u = { 'a' : 10, 'b' : 20, 'c' : [ 30,20] , 'd' : 40  }
u = d1 | d2 
i = d1 & d2 // intersection is and : i = { : } 
m12 = d1 - d2 // m12 = d1 
m21 = d2 - d1 // m12 = d2
delta = d1 ^ d2 // delta = {'a' : 10, 'b' : 20, 'd': 40}
\end{lstlisting}
\end{subsection}

\begin{subsection}{Collection Cross Product and Power}
The cross product, as defined, with the multiply operator:

\begin{lstlisting}[style=JexlStyle]
l1 = [1,2,3]
l2 = ['a','b' ]
cp = l1 * l2 
/*
[[1, a], [1, b], [2, a], [2, b], [3, a], [3, b]] := cp 
*/
\end{lstlisting}
Obviously we can think of power operation or exponentiation 
on a collection itself. That would be easy :

$$
A^0 := \{\} \; , \; A^1 := A \; , \; A^2 := A \times A
$$
and thus :
$$
A^n :=  A^{n-1} \times A
$$
For general collection power can not be negative.
Here are some examples now:

\begin{lstlisting}[style=JexlStyle]
b = [0,1]
gate_2 = b*b // [ [0,0],[0,1],[1,0],[1,1] ]
another_gate_2 = b ** 2 // same as b*b 
gate_3 = b ** 3 // well, all truth values for 3 input gate 
gate_4 = b ** 4 // all truth values for 4 input gate 
\end{lstlisting}

String is also a collection, and all of these are applicable for string too.
But it is a special collection, so only power operation is allowed.

\begin{lstlisting}[style=JexlStyle]
s = "Hello"
s2 = s**2 // "HelloHello"
s_1 = s**-1 // "olleH"
s_2 = s**-2 // "olleHolleH"
\end{lstlisting}
\end{subsection}

\begin{subsection}{Collection Relation Comparisons}
\index{Set Relations : Collection}

The operators are defined as such:
\begin{enumerate}
\item{ $A \subset B $ is defined as $A < B $  }
\item{ $A \subseteq B $ is defined as $A <= B $  }
\item{ $A \supset B $ is defined as $A > B $  }
\item{ $A \supseteq B $ is defined as $A >= B $  }
\item{ $A = B$ is defined as $A == B$  }
\end{enumerate}

Note that when collections can not be compared at all, 
it would return false to showcase that the relation fails.

So, we go again with sets:

\begin{lstlisting}[style=JexlStyle]
s1 = set(1,2,3)
s2 = set(1,3)
sub = s2 < s1 // true  
sup = s1 > s2 // true 
sube = ( s2 <= s1 ) // true  
supe = (s1 >= s2) // true 
s3 = set(5,6)
s1 < s3 // false 
s3 > s1 // false 
s1 != s3 // true
\end{lstlisting}

So, we go again with lists:

\begin{lstlisting}[style=JexlStyle]
l1 = list(1,2,3,3,4)
l2 = list(1,3,2)
sub = l2 < l1 // true  
sup = l1 > l2 // true 
sube = ( l2 <= l1 ) // true  
supe = (l1 >= l2) // true 
l3 = list(5,6)
l1 < l3 // false 
l3 > l1 // false 
l1 != l3 // true
\end{lstlisting}

And finally with dictionaries:

\begin{lstlisting}[style=JexlStyle]
d1 = {'a' : 10, 'b' : 20 , 'c' : 30 }
d2 = {'c' : 30 , 'a' : 10  }
sub = ( d2 < d1) // true 
sup = ( d1 > d2) // true
\end{lstlisting}
\end{subsection}

\begin{subsection}{Mixing Collections}
One can choose to intermingle $set$ with $list$, that promotes the $set$ to $list$.
Thus :

\begin{lstlisting}[style=JexlStyle]
s = set(1,2,3)
l = list(1,3,3,2)
sub = s < l  // true  
sup = l > s // true 
u = l | s // u = [1,2,3,3 ]
\end{lstlisting}

\end{subsection}


\begin{subsection}{Collections as Tuples}
\index{Set Operations : there exist : Collections}

When we say $[1,2] < [1,2,3,4] $ we obviously mean as collection itself.
But what about we start thinking as \href{https://en.wikipedia.org/wiki/Tuple}{tuples}?
Does a tuple contains in another tuple? That is we can find the items in order? 
"@" operator solves that too :

\begin{lstlisting}[style=JexlStyle]
l = [1,2]
m = [1,2,3,4]
in = l @ m // true 
\end{lstlisting}

The $index()$, and $rindex()$ generates the index of where the match occurred:

\begin{lstlisting}[style=JexlStyle]
l = [1,2]
m = [1,2,3,4,1,2]
// read which : index is l in m?
forward = index( l , m) // 0 
backward = rindex( l , m) // 4   
\end{lstlisting}

Sometimes it is of importance to get the notion of $starts\_with$ and $ends\_with$.
There are two special operators :
\index{Tuple Operation : Starts With, Ends With}

\begin{lstlisting}[style=JexlStyle]
// read m starts_with l 
sw = m #^ l // true 
// read m ends with l 
ew = m #$ l // true 
\end{lstlisting}

Note that this is also true for strings, as they are collections of characters.
So, these are legal:

\begin{lstlisting}[style=JexlStyle]
s = 'abracadabra'
prefix = 'abra'
suffix = prefix
i = index ( prefix, s ) // 0 
r = rindex( suffix, s ) // 7 
sw = s #^ prefix // true 
ew = s #$ suffix // true 
\end{lstlisting}

\end{subsection}

\end{section}


