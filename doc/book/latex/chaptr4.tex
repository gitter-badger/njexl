\chapter{Formal Constructs}\label{using-predicates}

{\LARGE F}ormalism is the last thing that stays in a software developers mind nowadays. 
This is a result of mismanaging expectation of how software and coding is taught and practiced in industry.
The majority view in the Industry is :
\begin{center}
\emph{Software is a form of art, and is not science at all}. 
\end{center}
It is easy to showcase that this is the prevailing feeling. When the last time people
actually did any mathematics to reach any conclusion like - what would be the optimal
object hierarchy size if any? What should be the optimal pool size for an application?
How one can optimally transform the data? 

Most of these cases, a bit of formal thinking solves a lot of the problems that can come due to bad design,
if there was a design in the first place (for most parts, there are none). 
That is, because software is nothing but formally applied computer science, and industry lacks computer science.
What can be done by 100 can also be done by 10 people, and no, they do not have to be really really smart,
they only have to be understanding the underlying structure. 
In reality software is :

\begin{center}
\emph{Software is to Applied Computation is what Architecture is to applied sciences, i.e. Engineering.}
\end{center}
Unless you have an idea of the underlying mechanism, most won't work,
and those who would luckily work - would simply fail in a slightly longer timeframe.
Moreover, user does not care about brilliant design and optimal code, user care about getting his/her problem 
solved, fast. It is business who should care about how to do things optimally and repeatedly, otherwise,
the business model won't sustain. 

This section would be specifically to understand what sort of formalism from computation 
we can use in practice to develop and test software to aid business. Because in the end, profit and revenue 
runs the show.

\begin{section}{Conditional Building Blocks}

\begin{subsection}{Formalism}
\index{Predicate Logic}
In a formal logic like \href{https://en.wikipedia.org/wiki/First-order\_logic}{FOPL}, 
the statements can be made from the basic ingredients, 
\emph{``there exist''} and \emph{``for all''}. We need to of course put some logical stuff like $AND$, $OR$,
and some arithmetic here and there, but that is what it is.

If we have noted down the \emph{sorting problem} in chapter 2, we would have restated the problem as :

\begin{center}
\emph{ There does not exist an element which is less than that of the previous one. }
\end{center} 

In mathematical logic, \emph{``there exists''} becomes : $\exists$ and \emph{``for all''} becomes $\forall$.
and logical not is shown as $\neg$ ,  So, the same formulation goes in : let 
$$
S = \{ 0, 1, 2, ... , size(a)-1  \} =  \{  x \in \mathbb{N} | x < size(a) \} 
$$
then :
$$
  i \in S ; \forall \; i  \neg \; ( \exists \; a[i] \; s.t. \; a[i] < a[i-1]    ) 
$$ 

And the precise formulation of what is a sorted array/list is done in the 
\href{https://en.wikipedia.org/wiki/Second-order_logic}{second order logic}.
The problem of programming and validation generally is expressive in terms of 
\href{https://en.wikipedia.org/wiki/Higher-order_logic}{higher order logic}.
\index{logic : 2nd order}\index{logic : higher order}
\end{subsection}
\begin{subsection}{Relation between ``there exists'' and ``for all'' }
\index{relation between exists and forall}
There is a nice dual relationship between there exists $\exists$ and $\forall$.
Given we have the negation operation defined, $\forall$ and $\exists$  are interchangeable.
Suppose we ask :  
\begin{center}
\emph{ Are every element in the collection C greater than 0? }
$$
\forall x \in C \; ; \; x > 0 \; ?
$$
\end{center}
This can be reformulated as the negation of:
\begin{center}
\emph{ Does there exist any element in the collection C less than or equal to 0? }
$$
\exists x \in C \; ; \; x \le 0 \; ?
$$
\end{center}
Hence, the transformation law is :

$$
\forall x \in C \; ; \; P(x) \; ? \Longleftrightarrow \neg ( \exists x \in C \; ; \; \neg P(x) \; ) ?
$$
\end{subsection}

\begin{subsection}{There Exist In Containers}
\index{Predicate Logic : there exists}
To check, if some element is, in some sense exists inside a container ( list, set, dict, array, heap )
one needs to use the $IN$ operator, which is $@$. \index{operator : at }
\index{ operator : division over dictionary }
\begin{center}\begin{minipage}{\linewidth}
\begin{lstlisting}[style=JexlStyle]
l = [1,2,3,4] // l is an array 
in = 1 @ l // in is true 
in = 10 @ l // in is false 
d = { 'a' : 10 , 'b' : 20  }
in = 'a' @ d // in is true 
in = 'c' @ d // in is false 
in = 10 @ d // in is false 
in = 10 @ d.values() // in is true 
/* division over a dictionary gives the keyset 
where the value of the key is the operand */
in = size( d / 10 ) > 0  // in is true 
s = "hello"
in = "o" @ s // in is true 
/* This works for linear collections even */
m = [2,3]
in = m @ l // true 
in = l @ l // true  
\end{lstlisting}
\end{minipage}\end{center}

This was not simply put in the place simply because we simply dislike $x.contains(y)$. 
In fact we do dislike the form of object orientation where there is no guarantee that $x$ would be null or not.
Worse, it is impossible to test for \emph{null} always, such is the prevailing nature of the bad code in Industry. 
Formally, then :

\begin{lstlisting}[style=JexlStyle]
// equivalent function 
def in_function(x,y){
   if ( empty(x) ) return false 
   return x.contains(y) 
}
// or one can use simply 
y @ x // same result
\end{lstlisting}

How about regular expressions? There are two operators related to regular expressions,
the \emph{match} operator, and then \emph{not match} operator. \index{regular expressions}
See the \href{http://www.regular-expressions.info/quickstart.html}{guide to regular expressions}.
\index{ operator : regex : match }
\index{ operator : regex : not match }

\begin{center}\begin{minipage}{\linewidth}
\begin{lstlisting}[style=JexlStyle]
re = "^[-+]?[0-9]*\.?[0-9]+([eE][-+]?[0-9]+)?$"
s = "hello"
match = ( s =~ re ) // false 
match = ( s !~ re ) // true 
f = "12.3456"
match = ( s =~ re ) // true 
match = ( s !~ re ) // false 
\end{lstlisting}
\end{minipage}\end{center}

Thus, the operator ``=\textasciitilde'' is the \emph{match} operator, while  ``!\textasciitilde'' is the \emph{not match} operator.
\index{ operator : \textasciitilde }

\end{subsection}

\begin{subsection}{Size of Containers : empty, size, cardinality}
To check whether a container is empty or not, 
use the $empty()$ \index{empty()} function, just mentioned above. Hence:
\begin{center}\begin{minipage}{\linewidth}
\begin{lstlisting}[style=JexlStyle]
n = null
e = empty(n) // true 
n = []
e = empty(n) // true 
n = list()
e = empty(n) // true 
d = { : }
e = empty(d) // true 
nn = [ null ]
e = empty(nn) // false 
\end{lstlisting}
\end{minipage}\end{center}

For the actual size of it, there are two alternatives.
One is the $size()$ \index{size()}  function :

\begin{center}\begin{minipage}{\linewidth}
\begin{lstlisting}[style=JexlStyle]
n = null
e = size(n) // -1 
n = []
e = size(n) // 0 
n = list()
e = size(n) // 0 
d = { : }
e = size(d) // 0 
nn = [ null ]
e = size(nn) // 1 
\end{lstlisting}
\end{minipage}\end{center}

Observe that it returns negative given input null. 
That is a very nice way of checking null.
The other one is the $cardinal$ operator \index{operator : cardinality }:

\begin{center}\begin{minipage}{\linewidth}
\begin{lstlisting}[style=JexlStyle]
n = null
e = #|n| // 0 
n = []
e = #|n| // 0 
n = list()
e = #|n| // 0 
d = { : }
e = #|d| // 0 
nn = [ null ]
e = #|nn| // 1 
\end{lstlisting}
\end{minipage}\end{center}

This operator in some sense gives a measure. It can not be negative, 
so cardinality of $null$ is also 0. 
\end{subsection}

\begin{subsection}{There Exist element with Condition : index, rindex }
Given we have if \emph{``y in container x''} or not,
what if when asked a question like : \emph{ ``is there a x in Collection C such that predicate P(x) is true''} ?
Formally, then, given a predicate $P(x)$, and a collection $C$, we ask :
$$
\exists x \in C \;s.t.\; P(x) = True
$$

This pose a delicate problem. 

Notice this is the same problem we asked about sorting. Is there an element `x' in $C$, 
such that the sorting order is violated? If not, the collection is sorted.
This brings back the $index$ function \index{index()}.
We are already familiar with the usage of $index()$ function from chapter 2.
But we would showcase some usage :
\index{Predicate Logic : there exist : index() }
 
\begin{lstlisting}[style=JexlStyle]
l = [ 1, 2, 3, 4, 5, 6 ]
// search an element such that double of it is 6
i = index{ $ * 2 == 6 }(l) // i : 2 
// search an element such that it is between 3 and 5 
i = index{ $ < 5 and $ > 3 }(l) // i : 3 
// search an element such that it is greater than 42
i = index{  $ > 42 }(l) // i : -1, failed 
\end{lstlisting}

The way index function operates is: the statements inside the anonymous block are executed.
If the result of the execution is true, the index function returns the index in the collection.
If for none of the elements the anonymous block assumes the true value, it returns a failure by returning $-1$.

NOTE that in the nJexl negative indices are proper indices into a collection.
Thus, a code like this is tantamount to be a disaster waiting to happen:
 
\begin{lstlisting}[style=JexlStyle]
l = [ 1, 2, 3, 4, 5, 6 ]
// search an element   
i = index{ $ > 42 }(l) // i : -1
// now I have found the element, so :
x = l[i] // no, x is 6. we must check the index value for >=0
\end{lstlisting}


Index function runs from left to right, and there is a variation $rindex()$ \index{index()}
which runs from right to left.

\index{Predicate Logic : there exist : rindex() }

\begin{lstlisting}[style=JexlStyle]
l = [ 1, 2, 3, 4, 5, 6 ]
// search an element such that it is greater than 3  
i = index{ $ > 3 }(l) // i : 3
// search an element such that it is greater than 3  
i = rindex{ $ > 3 }(l) // i : 5 
// search an element such that it is greater than 42
i = rindex{  $ > 42 }(l) // i : -1, failed 
\end{lstlisting}

Thus, the \emph{there exists} formalism is taken care by these operators and functions together.

\end{subsection}


\begin{subsection}{For All elements with Condition : select }
\index{Predicate Logic : for all : select() }

We need to solve the problem of \emph{for all}. This is done by $select()$ function \index{select()}.
The way select function works is : executes the anonymous statement block, and if the condition is true, 
then select and collect that particular element, and returns a list of collected elements.

\begin{lstlisting}[style=JexlStyle]
l = [ 1, 2, 3, 4, 5, 6 ]
// select all even elements 
evens = select{ $ % 2 == 0 }(l)
// select all odd elements 
odds = select{ $ % 2 == 1 }(l) 
\end{lstlisting}

\end{subsection}

\begin{subsection}{Partition a collection on Condition : partition }
Given a $select()$, we are effectively partitioning the collection into two halves,
$select()$ selects the matching partition. In case we want both the partitions, then 
we can use the $partition()$ function \index{partition()}.

\begin{center}\begin{minipage}{\linewidth}
\begin{lstlisting}[style=JexlStyle]
l = [ 1, 2, 3, 4, 5, 6 ]
#(evens,odds) = partition{ $ % 2 == 0 }(l)
write(evens) // prints 2, 4, 6 
write(odds) // prints 1, 3, 5
\end{lstlisting}\end{minipage}
\end{center}

\end{subsection}

\begin{subsection}{Collecting value on Condition : where }
Given a $select()$ or $partition()$, we are collecting the values already 
in the collection. What about we want to change the values? 
This is done by the conditional known as $where()$ \index{where()}.
The way where works is, we put the condition inside the where.
The result would be the result of the condition, but the where clause body 
would be get executed. Thus, we can change the value we want to collect
by replacing the $ITEM$ variable, that is $\$$.

\begin{center}\begin{minipage}{\linewidth}
\begin{lstlisting}[style=JexlStyle]
l = [ 1, 2, 3, 4, 5, 6 ]
#(evens,odds) = partition{ where($ % 2 == 0){ $ = $**2 } }(l)
write(evens) // prints 4, 16, 36 
write(odds) // prints 1, 3, 5
\end{lstlisting}\end{minipage}
\end{center}

This is also called \emph{item rewriting} \index{item rewriting}.
\end{subsection}

\begin{subsection}{Inversion of Logic}
As we have discussed there is a relation between $\exists$ and $\forall$, 
we can use them for practical applications. We would be using them everywhere, 
and as \href{http://psychology.about.com/od/psychologyquotes/a/lewinquotes.htm}{Kurt Lewin} said :
\begin{center}
\emph{There is nothing so practical as a good theory.}
\end{center}
So, we would start with all these examples in inverted logic.
Observe that $select()$ mandates a guaranteed \href{https://en.wikipedia.org/wiki/Big_O_notation}{runtime} of $\Theta(n)$,
while $index()$ has a probabilistic runtime of $\Theta(n/2)$. So, we should generally choose $index()$ over $select()$.
For example, take the problem of \emph{are all numbers in a list larger than 0}, it can be solve in two ways:

\begin{center}\begin{minipage}{\linewidth}
\begin{lstlisting}[style=JexlStyle]
l = [10,2,3,10, 2, 0 , 9 ]
// forall method 1
size(l) == size ( select{ $ > 0 }(l) )    
// forall method 2 : note the inversion of logic 
empty ( select{ $ <= 0 }(l) )    
// there exists method : note the inversion of condition from 1
( index{ $ <= 0 }(l) < 0 )
\end{lstlisting}\end{minipage}
\end{center}

Please choose the 3rd one, that makes sense, takes less memory, and is optimal.    
\end{subsection}

\end{section}

\begin{section}{Operators from Set Algebra}

\begin{subsection}{Set Algebra}
\index{Set Algebra}

\href{https://en.wikipedia.org/wiki/Algebra_of_sets}{Set algebra}
, in essence runs with the notions of the following ideas :
\begin{enumerate}
\item{There is an unique empty set.}
\item { The following operations are defined :
\begin{enumerate}{
     \item{Set Union is defined as : \index{Set Algebra : Union}
$$
U_{AB} = A \cup B \; :=  \{ x \in A \; or \; x \in B \} := U_{BA}
$$ 
}
    \item{Set Intersection is defined as : \index{Set Algebra : Intersection}
$$
I_{AB} = A \cap B \; :=  \{ x \in A \; and \; x \in B \} := I_{BA}
$$
}
\item{Set Minus is defined as : \index{Set Algebra : Minus}
$$
M_{AB} := A \setminus B \; :=  \{ x \in A \; and \; x \not \in B \} 
$$
thus, $M_{AB} \ne M_{BA}$. 
}
\item{Set Symmetric Difference is defined as : \index{Set Algebra : Symmetric Difference} 
$$
\Delta_{AB} := ( A \setminus B ) \cup ( B  \setminus A ) := \Delta_{BA}
$$
}
\item{Set Cross Product is defined as : 
$$
X_{AB} = \{ ( a, b ) \; | \; a \in A \; and \;  b \in B  \} 
$$
thus, $X_{AB} \ne X_{BA}$
}
}
\end{enumerate}}
\item{ The following relations are defined:
\begin{enumerate}

\item{ Subset Equals : \index{Set Algebra : Subset Equals}
$$
A \subseteq B \; when \;  x \in A \implies x \in B
$$
}

\item{ Equals : \index{Set Algebra : Equals}
$$
A = B \; when \; A \subseteq B \; and \; B \subseteq A
$$
}


\item{ Proper Subset : \index{Set Algebra : Subset,Proper}
$$
A \subset B \; when \; A \subseteq B \; and \; \exists x \in B \;s.t.\; x \not \in A
$$
}

\item{ Superset Equals: \index{Set Algebra : Superset Equals}
$$
A \supseteq B \; when \; B \subseteq A 
$$
}

\item{ Superset : \index{Set Algebra : Superset,Proper}
$$
A \supset B \; when \; B \subset A 
$$
}
\end{enumerate}}
\end{enumerate}
\end{subsection}

\begin{subsection}{Set Operations on Collections}
\index{Set Operations : Collection}

For the sets, the operations are trivial.

\begin{center}\begin{minipage}{\linewidth}
\begin{lstlisting}[style=JexlStyle]
s1 = set(1,2,3,4)
s2 = set(3,4,5,6)
u = s1 | s2 // union is or : u = { 1,2,3,4,5,6}
i = s1 & s2 // intersection is and : i = { 3,4 }
m12 = s1 - s2 // m12 ={1,2}
m21 = s2 - s1 // m12 ={5,6}
delta = s1 ^ s2 // delta = { 1,2,5,6}
\end{lstlisting}
\end{minipage}\end{center}


For the lists or arrays, where there can be multiple elements present, 
this means a newer formal operation.
Suppose in both the lists, an element `e' is present, in $n$ and $m$ times.
So, when we calculate the following :
\begin{enumerate}
\item{Intersection :
  the count of $e$ would be $min(n,m)$.
}
\item{Union :
  the count of $e$ would be $max(n,m)$.
}
\item{Minus :
  the count of $e$ would be $max(0,n-m)$.
}

\end{enumerate}
With this, we go on for lists:

\begin{center}\begin{minipage}{\linewidth}
\begin{lstlisting}[style=JexlStyle]
l1 = list(1,2,3,3,4,4)
l2 = list(3,4,5,6)
u = l1 | l2 // union is or : u = { 1,2,3,3,4,4,5,6}
i = l1 & l2 // intersection is and : i = { 3,4 }
m12 = l1 - l2 // m12 ={1,2,3,4}
m21 = l2 - l1 // m12 ={5,6}
delta = l1 ^ l2 // delta = { 1,2,3,4,5,6}
\end{lstlisting}
\end{minipage}\end{center}

Now, for dictionaries, the definition is same as lists, 
because there the dictionary can be treated as a list of key-value pairs.
So, for one pair to be equal to another, both the key and the value must match.
Thus:

\begin{center}\begin{minipage}{\linewidth}
\begin{lstlisting}[style=JexlStyle]
d1 = {'a' : 10, 'b' : 20 , 'c' : 30 }
d2 = {'c' : 20 , 'd' : 40  }
// union is or : u = { 'a' : 10, 'b' : 20, 'c' : [ 30,20] , 'd' : 40  }
u = d1 | d2 
i = d1 & d2 // intersection is and : i = { : } 
m12 = d1 - d2 // m12 = d1 
m21 = d2 - d1 // m12 = d2
delta = d1 ^ d2 // delta = {'a' : 10, 'b' : 20, 'd': 40}
\end{lstlisting}
\end{minipage}\end{center}

\end{subsection}

\begin{subsection}{Collection Cross Product and Power}
The cross product, as defined, with the multiply operator:

\begin{center}\begin{minipage}{\linewidth}
\begin{lstlisting}[style=JexlStyle]
l1 = [1,2,3]
l2 = ['a','b' ]
cp = l1 * l2 
/* [[1, a], [1, b], [2, a], [2, b], [3, a], [3, b]] := cp */
\end{lstlisting}
\end{minipage}\end{center}

Obviously we can think of power operation or exponentiation 
on a collection itself. That would be easy :

$$
A^0 := \{\} \; , \; A^1 := A \; , \; A^2 := A \times A
$$
and thus :
$$
A^n :=  A^{n-1} \times A
$$
For general collection power can not be negative.
Here are some examples now:

\begin{lstlisting}[style=JexlStyle]
b = [0,1]
gate_2 = b*b // [ [0,0],[0,1],[1,0],[1,1] ]
another_gate_2 = b ** 2 // same as b*b 
gate_3 = b ** 3 // well, all truth values for 3 input gate 
gate_4 = b ** 4 // all truth values for 4 input gate 
b ** 0 // [] : power zero is empty collection
\end{lstlisting}

String is also a collection, and all of these are applicable for string too.
But it is a special collection, so only power operation is allowed.

\begin{lstlisting}[style=JexlStyle]
s = "Hello"
s2 = s**2 // "HelloHello"
s_1 = s**-1 // "olleH"
s_2 = s**-2 // "olleHolleH"
\end{lstlisting}

One interesting case is that of negative power, which simply reverses a collection, 
if such thing is possible : \index{collections : reverse}

\begin{lstlisting}[style=JexlStyle]
b = [0,1]
rb = b** -1 // [1 , 0 ]
s = "hello" 
rs = s**-1 // "olleh" 
\end{lstlisting}

A negative power is a valid power, thus:

\begin{lstlisting}[style=JexlStyle]
b = [0,1]
rb = b** -2 // [[1, 0], [1, 1], [0, 0], [0, 1]]
s = "hello" 
rs = s**-2 // "olleholleh" 
\end{lstlisting}

these are simply not there, in the examples section
we will see how they ease out some interesting algorithms.
A very practical use case of power operator for string
is padding by zeros for a binary integer of a fixed size:

\begin{lstlisting}[style=JexlStyle]
n = 23
bn = str(n,2) // convert into binary string
padded_8 = "0" ** (8 - #|bn| ) + bn // 00010111 : convert to 8 bit 
\end{lstlisting}

\end{subsection}

\begin{subsection}{Collection Relation Comparisons}
\index{Set Relations : Collection}

The operators are defined as such:
\begin{enumerate}
\item{ $A \subset B $ is defined as $A < B $  }
\item{ $A \subseteq B $ is defined as $A <= B $  }
\item{ $A \supset B $ is defined as $A > B $  }
\item{ $A \supseteq B $ is defined as $A >= B $  }
\item{ $A = B$ is defined as $A == B$  }
\end{enumerate}

Note that when collections can not be compared at all, 
it would return false to showcase that the relation fails.

So, we go again with sets:

\begin{center}\begin{minipage}{\linewidth}
\begin{lstlisting}[style=JexlStyle]
s1 = set(1,2,3)
s2 = set(1,3)
sub = s2 < s1 // true  
sup = s1 > s2 // true 
sube = ( s2 <= s1 ) // true  
supe = (s1 >= s2) // true 
s3 = set(5,6)
s1 < s3 // false 
s3 > s1 // false 
s1 != s3 // true
\end{lstlisting}
\end{minipage}\end{center}

So, we go again with lists:

\begin{center}\begin{minipage}{\linewidth}
\begin{lstlisting}[style=JexlStyle]
l1 = list(1,2,3,3,4)
l2 = list(1,3,2)
sub = l2 < l1 // true  
sup = l1 > l2 // true 
sube = ( l2 <= l1 ) // true  
supe = (l1 >= l2) // true 
l3 = list(5,6)
l1 < l3 // false 
l3 > l1 // false 
l1 != l3 // true
\end{lstlisting}
\end{minipage}\end{center}


And finally with dictionaries:

\begin{center}\begin{minipage}{\linewidth}
\begin{lstlisting}[style=JexlStyle]
d1 = {'a' : 10, 'b' : 20 , 'c' : 30 }
d2 = {'c' : 30 , 'a' : 10  }
sub = ( d2 < d1) // true 
sup = ( d1 > d2) // true
\end{lstlisting}
\end{minipage}\end{center}

\end{subsection}


\begin{subsection}{Mixing Collections}
One can choose to intermingle $set$ with $list$, that promotes the $set$ to $list$.
Thus :

\begin{center}\begin{minipage}{\linewidth}
\begin{lstlisting}[style=JexlStyle]
s = set(1,2,3)
l = list(1,3,3,2)
sub = s < l  // true  
sup = l > s // true 
s == l // false : promotes both to list and checks
u = l | s // u = [1,2,3,3 ]
\end{lstlisting}
\end{minipage}\end{center}

\end{subsection}


\begin{subsection}{Collections as Tuples}
\index{Set Operations : there exist : Collections}

When we say $[1,2] < [1,2,3,4] $ we obviously mean as collection itself.
But what about we start thinking as \href{https://en.wikipedia.org/wiki/Tuple}{tuples}?
Does a tuple contains in another tuple? That is we can find the items in order? 
"@" operator solves that too :

\begin{center}\begin{minipage}{\linewidth}
\begin{lstlisting}[style=JexlStyle]
l = [1,2]
m = [1,2,3,4]
in = l @ m // true 
\end{lstlisting}
\end{minipage}\end{center}

The $index()$, and $rindex()$ generates the index of where the match occurred:

\begin{lstlisting}[style=JexlStyle]
l = [1,2]
m = [1,2,3,4,1,2]
// read which : index is l in m?
forward = index( l , m) // 0 
backward = rindex( l , m) // 4   
\end{lstlisting}

Sometimes it is of importance to get the notion of $starts\_with$ and $ends\_with$.
There are two special operators :
\index{Tuple Operation : Starts With, Ends With}

\begin{lstlisting}[style=JexlStyle]
// read m starts_with l 
sw = m #^ l // true 
// read m ends with l 
ew = m #$ l // true 
\end{lstlisting}

Note that this is also true for strings, as they are collections of characters.
So, these are legal:

\begin{lstlisting}[style=JexlStyle]
s = 'abracadabra'
prefix = 'abra'
suffix = prefix
i = index ( prefix, s ) // 0 
r = rindex( suffix, s ) // 7 
sw = s #^ prefix // true 
ew = s #$ suffix // true 
\end{lstlisting}

These are not fancy stuffs. These are necessary to move over from the stupidity that is known 
as \emph{null based programming} where lots of efforts gets nullified by the stupidity of null.
Observe, the same null thing:

\begin{lstlisting}[style=JexlStyle]
s = 'abracadabra'
prefix = 'abra'
null #^ s // false 
prefix #^ null // false 
[null,null] #^ null // true 
\end{lstlisting}

and thus, we can see where it does the value add.
Python fails in this case, because none of it's syntaxes permits None type
to be used as collection to yield True/False for \emph{in} or \emph{matches}.  

There is this other operator called \emph{InOrder : \#@ } \index{collections : in order} which
tells whether a sub collection exists in order, as tuple in the larger collection 
when the larger collection is considered as tuple:

\begin{lstlisting}[style=JexlStyle]
l = [0,1,2,3,4]
[1,2] #@ l // true 
[2] #@ l // true 
[2,1] #@ l // false 
\end{lstlisting}
as usual, they are all null ready.    
The \emph{in : @} operator is different from \emph{in order},
because clearly :

\begin{lstlisting}[style=JexlStyle]
l = [0,1,2,3,4]
[2,1] #@ l // false 
[2,1] @ l // true 
\end{lstlisting}
 
Thus, in order is precisely what it is, if the items from the passed collection
occurs as sub tuples ( can be build by projection operations ) of the parent collection. 
Formally, let's end this with a bit of formalism as we have started it.

Define a projection operation $\Pi_{i,j}(C) $ with $i \le j $ which slice the collection $C$
from index $i$ to index $j$, or rather a \emph{sub()} function. 
The following are defined as such :

\begin{enumerate}
\item{
  \textbf{ In Order :} \\
   $ x \; \#@ \; C $ iff $\exists (i,j)$ such that $\Pi_{i,j}(C) = x $.  
}
\item{
  \textbf{ Starts With :} \\
   $ C \; \# \textasciicircum \; x $ iff $\exists j$ such that $\Pi_{0,j}(C) = x $.  
}
\item{
  \textbf{ Ends With :} \\
   $ C \; \#\$ \; x $ iff $\exists i$ such that $\Pi_{i,|C|-1}(C) = x $,
   where $|C|$ is the cardinality of collection C.
}
\end{enumerate}
\end{subsection}

\begin{subsection}{Finding the Relation between Collections}
Sometimes it is necessary to find relation between collections.
Instead of going about it in a round about way, nJexl lets you do so
in a genuine manner, by using the \emph{SetOperations} class.
Observe when \emph{set:} is aliased to the class:
\index{ collection : set\_relation() }
\begin{lstlisting}[style=JexlStyle]
A = [0,1,2,3,4]
B = [1,2,4]
C = [4,6,9]
D = [0,2,3]
// treat each collection as sets
r1 = set:set_relation(B,A) // SUBSET
r2 = set:set_relation(A,B) // SUPERSET
r3 = set:set_relation(B,C) // OVERLAP
r4 = set:set_relation(D,C) // INDEPENDENT
// thus due to conversation of both the inputs to sets...
r5 = set:set_relation([0,2,2],[2,0]) // EQUAL !!!
\end{lstlisting}
Hence, to avoid such issues use \emph{list\_relation} method :
\index{ collection : list\_relation() }
\begin{lstlisting}[style=JexlStyle]
// treat each collection as lists
r5 = set:list_relation([0,2,2],[2,0]) // SUPERSET
r6 = set:list_relation([2,0],[0,2,2]) // SUBSET
\end{lstlisting}


\end{subsection}



\end{section}


