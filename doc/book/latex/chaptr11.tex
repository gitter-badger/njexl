\chapter{Java Connectivity}\label{java-connectivity}

{\LARGE J}VM is the powering force behind nJexl.
This chapter illustrates various usage scenarios where nJexl can be
embedded inside a plain Java Program.

\begin{section}{How to Embed}

\begin{subsection}{Dependencies}
\index{maven}
In the dependency section (latest release is 0.2 ) : 
\begin{lstlisting}[style=XmlStyle]
<dependency>
  <groupId>com.github.nmondal</groupId>
  <artifactId>njexl.lang</artifactId>
  <version>0.3-ALPHA-1</version> <!-- or 0.3-SNAPSHOT -->
</dependency>
\end{lstlisting}
That should immediately make your project a nJexl supported one. 
\end{subsection}

\begin{subsection}{Programming Model}
\index{ java connectivity : programming model }
nJexl has a random access memory model.
The script iterpreter is generally single threaded,
but each thread is given its own engine to avoid
the interpreter problem of \href{https://en.wikipedia.org/wiki/Global\_interpreter\_lock}{GIL}.

The memory comes in two varieties, default registers, which are used for global variables.
There is also this purely abstract 
\href{https://github.com/nmondal/njexl/blob/master/lang/src/main/java/com/noga/njexl/lang/JexlContext.java}{JexlContext} type, 
which abstracts how local storage gets used. This design makes nJexl remarkably memory heavy,
but at the same time pretty neat to integrate with any other JVM language, specifically Java. 
Any time a new scope is created, the $copy()$ mechanism ensures that the parent context is suitably passed along.
This is what really makes nJexl memory heavy, but fast in terms of access time of variables.

The nJexl \href{https://github.com/nmondal/njexl/blob/master/lang/src/main/java/com/noga/njexl/lang/JexlEngine.java}{JexlEngine} 
creates an \href{https://github.com/nmondal/njexl/blob/master/lang/src/main/java/com/noga/njexl/lang/Expression.java}{Expression} 
or a \href{https://github.com/nmondal/njexl/blob/master/lang/src/main/java/com/noga/njexl/lang/Script.java}{Script} 
from a string or a file. 

An interpreter is created when one either $evaluate()$ the Expression or $execute()$ the script.
To use variables, one must pass a $JexlContext$ in the evaluate or execute function.   

The result of these calls will always be an object. In case somehow the call does not return 
any value, it would return a $null$ value. 

\end{subsection}

\begin{subsection}{Example Embedding}
One sample embedding is furnished here :
\index{ java connectivity : sample embedding }


\begin{center}\begin{minipage}{\linewidth}
\begin{lstlisting}[style=myJavaStyle]
import com.noga.njexl.lang.JexlContext;
import com.noga.njexl.lang.JexlEngine;
import com.noga.njexl.lang.JexlException;
import com.noga.njexl.lang.Script;
import java.util.*;

public class Main {
   public static void main(String[] args){
        // get an already nicely prepared JexlContext
        JexlContext jc = com.noga.njexl.lang.Main.getContext();
        // get a list of functions which would become 
        // already imported namespaces
        Map<String, Object> functions 
            = com.noga.njexl.lang.Main.getFunction(jc);
        // create an Engine 
        JexlEngine JEXL = com.noga.njexl.lang.Main.getJexl(jc);
        // set the functions 
        JEXL.setFunctions(functions);
        // whatever the arguments are, add to the variables 
        jc.set(Script.ARGS, args);
        try {
            // import the script -- always use this new one 
            Script sc = JEXL.importScript(args[0]);
            // execute the script 
            Object o = sc.execute(jc);
            int e = 0;
            if (o instanceof Integer) {
                e = (int) o;
            }
            System.exit(e);
        } catch (Throwable e) {
            System.exit(1);
        }
   }
}
\end{lstlisting}  
\end{minipage}\end{center}
\end{subsection}

\begin{subsection}{Default Namespaces}
The standard nJexl core comes with the default namespaces :
\index{ java connectivity : adding to namespaces }

\begin{enumerate}
\item{ sys : 'java.lang.System' }
\item{ set :  
  \href{https://github.com/nmondal/njexl/blob/master/lang/src/main/java/com/noga/njexl/lang/extension/SetOperations.java}{SetOperations} 
}
\item{ str : 'java.lang.String' }
\item{ math : 'java.lang.Math' }
\end{enumerate}

These are set using the $setFunctions()$ call :

\begin{lstlisting}[style=myJavaStyle]
JEXL.setFunctions(functions);
\end{lstlisting} 

\end{subsection}

\begin{subsection}{Default Imports}
\index{ java connectivity : default imports }

To ensure that the import happens, we need to 
fix two things :
\begin{enumerate}
\item{Add a namespace that is same as the import id. }
\item{Add a variable that is same as the import id. }
\end{enumerate}

Doing only the first one will only add the namespace,
while doing only the second one would only add a local 
variable with the name of the import id.

\begin{lstlisting}[style=myJavaStyle]
Map f = JEXL.getFunctions();
// add the namespace 
f.set('my_id' , my_obj);
// add the variable to the context 
jc.set('my_id' , my_obj);
// set the functions map back 
JEXL.setFunctions(f);
// import happened now 
\end{lstlisting} 


\end{subsection}

\end{section}

\begin{section}{Programming nJexl}

\begin{subsection}{Script Object Model}
\index{ java connectivity : Script Object Model }


A script object holds the classes defined in it, 
as well as the \href{https://github.com/nmondal/njexl/blob/master/lang/src/main/java/com/noga/njexl/lang/extension/oop/ScriptMethod.java}{ScriptMethod} objects. So, after creating a script, one can access the ScriptMethods by simply calling $methods()$
function on the Script :

\begin{lstlisting}[style=myJavaStyle]
Script sc = JEXL.importScript("script.jxl");
Map<String,ScriptMethod> methods =  sc.methods();
\end{lstlisting} 

\end{subsection}

\begin{subsection}{Function Model}
\index{ java connectivity : Function Model }

Once we have the functions map now, we can get a function by name :

\begin{lstlisting}[style=myJavaStyle]
Script sc = JEXL.importScript("script.jxl");
ScriptMethod m = methods.get("my_function") ;
\end{lstlisting} 

However, to call this method one needs to pass an 
\href{https://github.com/nmondal/njexl/blob/master/lang/src/main/java/com/noga/njexl/lang/Interpreter.java}{Interpreter} :

\begin{lstlisting}[style=myJavaStyle]
Interpreter i = new Interpreter(
      JEXL, jc, strictFlag, silentFlag) ;
Object result = m.invoke(null,i,
                   new Object[]{ "These" , "are", "args" } );
\end{lstlisting} 

And thus, you can execute any nJexl methods from Java.
\end{subsection}

\end{section}

\begin{section}{Extending nJexl}

There are two extension points for Java developers
who are going to make their classes and functions nJexl ready.
 
\begin{subsection}{Overloading Operators}
\index{ java connectivity : operator overloading }

First of them is the overloading of operators.
The operators are defined in 
\href{https://github.com/nmondal/njexl/blob/master/lang/src/main/java/com/noga/njexl/lang/extension/oop/ScriptClassBehaviour.java}{ScriptClassBehaviour}. The specific operators are to be found under : $Arithmetic$ and $Logic$.

Any class implementing these interfaces would be used by the nJexl engine as if the operators are
overloaded for them. Thus, one can create Java objects which mixes up with objects of nJexl.
Most of the time it won't be necessary, but if it is, there is a way to achieve it.

For example, one can create a complex number class, which would implement Arithmetic.
\end{subsection}

\begin{subsection}{Implementing Named Arguments}
\index{ java connectivity : functions : named argument}

If one wants to implement a method which would mix properly with nJexl named arguments,
then one must expect a specific object $NamedArgs$ found in the Interpreter.java.

\begin{lstlisting}[style=myJavaStyle]
/* name of the argument */
public final String name;
/* the value that was passed */
public final Object value;
\end{lstlisting}

Thus, once we know that the argument is $NamedArgs$, we can extract the value, 
and do the due diligence. 

\end{subsection}


\begin{subsection}{Using Anonymous Arguments}
\index{ java connectivity : function : anonymous argument }

If one wants to implement the anonymous blocks like almost all the nJexl functions do, 
one needs to expect the first argument to be an $AnonymousParam$ which is also found in the Interpreter.java.
The standard way to process the argument can be said as :

\begin{center}\begin{minipage}{\linewidth}
\begin{lstlisting}[style=myJavaStyle]
/* ensure that the functions are always multiple args - nJexl default */
public static Obejct some_awesome_function(Object... args){
  AnonymousParam anon = null;
  if ( args.length == 0 ) { /* due diligence */  return null;  }
  if (args[0] instanceof AnonymousParam) {
            anon = (AnonymousParam) args[0];
            args = TypeUtility.shiftArrayLeft(args, 1);
  }
  // now in here, we may or may not have the anon set:
  if ( anon == null ){
     // try on with the normal way 
     return someThing; 
  }
  // in here I have anonymous param :
  Object context;
  Object item;
  Object id;
  Collection partial;
  anon.setIterationContextWithPartial(context, item, id , partial);
  Object o;
   try {
       o = anon.execute();
       if (o instanceof JexlException.Continue) {
            if (((JexlException.Continue) o).hasValue) {
                partial.add(((JexlException.Continue) o).value);
              }
               continue;
        }
        if (o instanceof JexlException.Break) {
             if (((JexlException.Break) o).hasValue) {
                  partial.add(((JexlException.Break) o).value);
               }
                    break;
         }
         // in here, use o :
         doSomethingWith(o);
        } catch (Exception e) {
                o = null;
     }
}
\end{lstlisting}
\end{minipage}\end{center}

\end{subsection}


\end{section}
