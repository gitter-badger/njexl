\chapter{Types and Conversions}\label{type-conversions}

{\LARGE T}ypes are not much useful for general purpose programming, save that they avoid errors.
Sometimes they are necessary, and some types are indeed useful because they let us do 
many useful stuff. In this chapter we shall talk about these types and how to convert
one to another.

The general form for type casting can be written :
\begin{lstlisting}[style=JexlStyle]
val = type_function(value, optional_default_value = null )
\end{lstlisting}

How does this work? The system would try to cast $value$
into the specific type. If it failed, and there is no default value, 
it would return $null$. However, if default is passed, it would return 
that when conversion fails.
This neat design saves a tonnage of $try \; ... \; catch$ stuff. 

\begin{section}{Integer Family}
This section is dedicated to the natural numbers family.
We have :
\begin{enumerate}
\item{bool : \href{https://docs.oracle.com/javase/8/docs/api/java/lang/Boolean.html}{Boolean} }
\item{short : \href{https://docs.oracle.com/javase/8/docs/api/java/lang/Short.html}{Short} }
\item{char :  \href{https://docs.oracle.com/javase/8/docs/api/java/lang/Character.html}{Character} }
\item{int : \href{https://docs.oracle.com/javase/8/docs/api/java/lang/Integer.html}{Integer} }
\item{long : \href{https://docs.oracle.com/javase/8/docs/api/java/lang/Long.html}{Long} }
\item{INT : \href{https://docs.oracle.com/javase/8/docs/api/java/math/BigInteger.html}{BigInteger} }
\item{Z : The natural Integer type, picks the smallest container from int,long, BigInteger}
\end{enumerate}
There are no \href{https://docs.oracle.com/javase/tutorial/java/nutsandbolts/datatypes.html}{primitive types}. 
That was an absolutely disaster of a design that was made in Java land, which I fixed it here.
Everything is, under the hood, an object here. 

\begin{subsection}{Boolean}\index{bool()}
The syntax is :
\begin{lstlisting}[style=JexlStyle]
val = bool(value, optional_default_value = null )
val = bool(value, optional_matching_values[2])
\end{lstlisting}
Observe both in action :
\begin{center}\begin{minipage}{\linewidth}
\begin{lstlisting}[style=JexlStyle]
val = bool("hello") // val is null
val = bool("hello",false) // val is false
val = bool('hi', ['hi' , 'bye' ] ) // val is true 
val = bool('bye', ['hi' , 'bye' ] ) // val is false 
\end{lstlisting}
\end{minipage}\end{center}

\end{subsection}

\begin{subsection}{Short}\index{short()}
This is almost never required, and is there for backward compatibility with Java data types.
The syntax is :

\begin{lstlisting}[style=JexlStyle]
val = short(value, optional_default_value = null )
\end{lstlisting}

Usage is :

\begin{lstlisting}[style=JexlStyle]
val = short("hello") // val is null
val = short("hello",0) // val is 0
val = short('42') // val is 42 
val = short(42) // val is 42 
\end{lstlisting}

\end{subsection}

\begin{subsection}{Character}\index{char()}
For almost all practical purposes, character is noting but the short value, 
interpreted by a code page.

The syntax is :

\begin{lstlisting}[style=JexlStyle]
val = char(value, optional_default_value = null )
\end{lstlisting}

Usage is :

\begin{lstlisting}[style=JexlStyle]
val = char("hello") // val is "h"
val = char(121231231,0) // val is 0
val = char('4') // val is '4' 
// ascii stuff?
val = char(65) // val is 'A'  
\end{lstlisting}

Generally none needs to get into this, because generally $String.charAt(index)$ is a good substitute
for finding a character.
\end{subsection}

\begin{subsection}{Integer}\index{int()}
This is very useful and the syntax is :

\begin{lstlisting}[style=JexlStyle]
val = int(value, optional_default_value = null )
\end{lstlisting}

Usage is :

\begin{lstlisting}[style=JexlStyle]
val = int("hello") // val is null
val = int("hello",0) // val is 0
val = int('42') // val is 42 
val = int(42) // val is 42 
val = int ( 10.1 ) // val is 10 
val = int ( 10.9 ) // val is 10 
\end{lstlisting}

\end{subsection}


\begin{subsection}{Long}\index{long()}
This is rarely required, and the syntax is :

\begin{lstlisting}[style=JexlStyle]
val = long(value, optional_default_value = null )
\end{lstlisting}

Usage is :

\begin{lstlisting}[style=JexlStyle]
val = long("hello") // val is null
val = long("hello",0) // val is 0
val = long('42') // val is 42 
val = long(42) // val is 42 
val = long( 10.1 ) // val is 10 
val = long( 10.9 ) // val is 10 
\end{lstlisting}

\end{subsection}

\begin{subsection}{BigInteger}\index{INT()}
This is sometimes required, and the syntax is :

\begin{lstlisting}[style=JexlStyle]
val = INT(value, base=10, default_value = null )
\end{lstlisting}
Usage is :

\begin{lstlisting}[style=JexlStyle]
val = INT("hello") // val is null
val = INT('hi',10,42 )// val is 42
val = INT('42') // val is 42 
val = INT(54,13 ) // val is 42 
\end{lstlisting}
\end{subsection}

\begin{subsection}{Z}\index{Z()}
This is required, when you want to simply
forget about everything and wants to typecast
an object to integer family. 
The syntax is :

\begin{lstlisting}[style=JexlStyle]
val = Z(value, base=10, default_value = null )
\end{lstlisting}

Usage is :

\begin{center}\begin{minipage}{\linewidth}
\begin{lstlisting}[style=JexlStyle]
val = Z("hello") // val is null
val = Z('hi',10,42 )// val is 42 int
val = Z('42121131231313') // val is long 
val = Z(54,13 ) // val is 42 
val = Z('4212113123131313131121323131') // val is BigInteger 
\end{lstlisting}
\end{minipage}\end{center}

\end{subsection}


\end{section}


\begin{section}{Rational Numbers Family}
This section is dedicated to the floating point numbers family.
We have :
\begin{enumerate}
\item{float : \href{https://docs.oracle.com/javase/8/docs/api/java/lang/Float.html}{Float} }
\item{double : \href{https://docs.oracle.com/javase/8/docs/api/java/lang/Double.html}{Double}}
\item{BigDecimal : \href{https://docs.oracle.com/javase/8/docs/api/java/math/BigDecimal.html}{BigDecimal} }
\item{Q : Picks the smallest container between float, double, BigDecimal }
\end{enumerate}

\begin{subsection}{Float}\index{float()}
This is not very useful and the syntax is :

\begin{lstlisting}[style=JexlStyle]
val = float(value, optional_default_value = null )
\end{lstlisting}

Usage is :

\begin{lstlisting}[style=JexlStyle]
val = float("hello") // val is null
val = float("hello",0) // val is 0.0
val = float('42') // val is 42.0 
val = float(42) // val is 42.0 
val = float ( 10.1 ) // val is 10.1 
val = float ( 10.9 ) // val is 10.9 
\end{lstlisting}
Note that, nJexl will automatically shrink floating point data into float, 
given it can fit the precision in :

\begin{lstlisting}[style=JexlStyle]
val = 0.01 // val is a float type, automatic
\end{lstlisting}


\end{subsection}


\begin{subsection}{Double}\index{double()}
This is generally required, and the syntax is :

\begin{lstlisting}[style=JexlStyle]
val = double(value, optional_default_value = null )
\end{lstlisting}

Usage is :

\begin{center}\begin{minipage}{\linewidth}
\begin{lstlisting}[style=JexlStyle]
val = double("hello") // val is null
val = double("hello",0) // val is 0.0
val = double('42') // val is 42.0 
val = double(42) // val is 42.0 
val = double( 10.1 ) // val is 10.1 
val = double( 10.9 ) // val is 10.9 
\end{lstlisting}
\end{minipage}\end{center}


\end{subsection}

\begin{subsection}{BigDecimal}\index{DEC()}
This is sometimes required, and the syntax is :

\begin{lstlisting}[style=JexlStyle]
val = DEC(value,default_value = null )
\end{lstlisting}

Usage is :
\begin{center}\begin{minipage}{\linewidth}
\begin{lstlisting}[style=JexlStyle]
val = DEC("hello") // val is null
val = DEC('hi', 42 )// val is 42.0
val = DEC('42') // val is 42.0 
val = DEC(42.00001 ) // val is 42.00001
\end{lstlisting}
\end{minipage}\end{center}
\end{subsection}

\begin{subsection}{Q}\index{Q()}
This is required, when you want to cast
an object into a floating point rational number, 
and the syntax is :

\begin{lstlisting}[style=JexlStyle]
val = Q(value,default_value = null )
\end{lstlisting}

Usage is :
\begin{center}\begin{minipage}{\linewidth}
\begin{lstlisting}[style=JexlStyle]
val = Q("hello") // val is null
val = Q('hi', 42 )// val is 42.0
val = Q('42') // val is 42.0 
val = Q(42.00001 ) // val is 42.00001 float 
val = Q(42.000014201014 ) // val is double 
val = Q(42.00001420101410801980981092101 ) // val is BigDecimal 
\end{lstlisting}
\end{minipage}\end{center}
\end{subsection}
\end{section}

\begin{section}{Generic Numeric Conversion : NUM()}
\index{NUM()}\index{generic numeric conversion}
For generic numeric conversions, there is \emph{NUM()} function,
whose job is to convert data types into proper numeric type, 
with least storage. Thus :

\begin{lstlisting}[style=JexlStyle]
val = NUM(value,default_value)
\end{lstlisting}

Usage is :
\begin{center}\begin{minipage}{\linewidth}
\begin{lstlisting}[style=JexlStyle]
val = NUM("hello") // val is null
val = NUM('hi', 42 )// val is 42 int
val = NUM('42.00') // val is 42 int 
val = NUM(42.00001 ) // val is 42.00001 float 
val = NUM(42.000014201014 ) // val is double 
// val is BigDecimal
val = NUM(42.00001420101410801980981092101 )  
\end{lstlisting}
\end{minipage}\end{center}


\end{section}


\begin{section}{The Chrono Family}
\index{date}\index{time}\index{instant}\index{format : date and time}

Handling date and time has been a problem, that too with timezones.
nJexl simplifies the stuff. 
We have three basic types to handle date/time:
\begin{enumerate}
\item{date : 
\href{https://docs.oracle.com/javase/8/docs/api/java/util/Date.html}{java.util.Date} : because of Java compatibility with SDKs. }
\item{time : 
\href{http://www.joda.org/joda-time/apidocs/org/joda/time/DateTime.html}{org.joda.time.DateTime} : included because it is the best chrono library out there. }
\item{instant : 
\href{https://docs.oracle.com/javase/8/docs/api/java/time/Instant.html}{java.time.Instant} : for newer systems who wants to experiment. }
\end{enumerate}


\begin{subsection}{Date}\index{date()}
This is how you create a date:
\begin{lstlisting}[style=JexlStyle]
val = date([ value, date_format ] )
\end{lstlisting}

With no arguments, it gives the current date time:

\begin{lstlisting}[style=JexlStyle]
today = date()
\end{lstlisting}

The default date format is $yyyyMMdd$, so :
\begin{lstlisting}[style=JexlStyle]
dt = date('20160218') // Thu Feb 18 00:00:00 IST 2016
\end{lstlisting}

For all the date formats on dates which are supported, 
see \href{https://docs.oracle.com/javase/8/docs/api/java/text/SimpleDateFormat.html}{SimpleDateFormat}.

Take for example :

\begin{lstlisting}[style=JexlStyle]
dt = date('2016/02/18', 'yyyy/MM/dd' ) 
// dt := Thu Feb 18 00:00:00 IST 2016
dt = date('2016-02-18', 'yyyy-MM-dd' ) 
// dt := Thu Feb 18 00:00:00 IST 2016
\end{lstlisting}

\end{subsection}

\begin{subsection}{Time}\index{time()}

This is how you create a joda \href{http://joda-time.sourceforge.net/apidocs/org/joda/time/DateTime.html}{DateTime}:
\begin{lstlisting}[style=JexlStyle]
val = time([ value, date_format , time_zone] )
\end{lstlisting}

With no arguments, it gives the current date time:

\begin{lstlisting}[style=JexlStyle]
today = time()
\end{lstlisting}

The default date format is $yyyyMMdd$, so :
\begin{lstlisting}[style=JexlStyle]
dt = time('20160218') // 2016-02-18T00:00:00.000+05:30
\end{lstlisting}

For all the date formats on dates which are supported, 
see \href{http://www.joda.org/joda-time/apidocs/org/joda/time/format/DateTimeFormat.html}{DateTimeFormat}.

Take for example :

\begin{lstlisting}[style=JexlStyle]
dt = time('2016/02/18', 'yyyy/MM/dd' ) 
// dt := 2016-02-18T00:00:00.000+05:30
dt = time('2016-02-18', 'yyyy-MM-dd' ) 
// dt := 2016-02-18T00:00:00.000+05:30
\end{lstlisting}

If you want to convert one time to another timezone, you need to give the time, 
and the \href{http://joda-time.sourceforge.net/timezones.html}{timezone}:

\begin{lstlisting}[style=JexlStyle]
dt = time()
dt_honolulu = time(dt , 'Pacific/Honolulu' ) 
// dt_honolulu := 2016-02-17T17:23:02.754-10:00 
dt_ny = time( dt, 'America/New_York' ) 
// dt_ny := 2016-02-17T22:23:02.754-05:00
\end{lstlisting}

\end{subsection}


\begin{subsection}{Instant}\index{instant()}

With no arguments, it gives the current instant time:

\begin{lstlisting}[style=JexlStyle]
today = instant()
\end{lstlisting}

It is freely mixable with other chrono types.

\end{subsection}


\begin{subsection}{Comparison on Chronos}\index{operator : compare, chrono }
All these date time types are freely mixable, 
and all comparison operations are defined with them.
Thus :

\begin{lstlisting}[style=JexlStyle]
d = date() // wait for some time 
t = time() // wait for some time 
i = instant()
// now compare 
c = ( d < t and t < i ) // true 
c = ( i > t and t > d ) // true 
\end{lstlisting}

Tow dates can be equal to one another, but not two instances,
that is a very low probability event, almost never.
Thus, equality makes sense when we know it is date, and not instant :

\begin{center}\begin{minipage}{\linewidth}
\begin{lstlisting}[style=JexlStyle]
d = date('19470815') // Indian Day of Independence  
t = time('19470815') // Indian Day of Independence  
// now compare 
c = ( d == t ) // true 
\end{lstlisting}
\end{minipage}\end{center}

\end{subsection}

\begin{subsection}{Arithmetic on Chronos}
\index{arithmetic : chorono}
Dates, obviously can not be multiplied or divided.
That would be a sin. However, dates can be added with other reasoanable values, 
dates can be subtracted from one another, and $time()$ can be added or subtracted by days, months or even year.
More of what all date time supports, 
see the manual of \href{http://joda-time.sourceforge.net/apidocs/org/joda/time/DateTime.html}{DateTime}.

To add time to a date, there is another nifty method :

\begin{lstlisting}[style=JexlStyle]
d = date('19470815') // Indian Day of Independence  
time_delta_in_millis = 24 * 60 * 60 * 1000 // next day 
nd = date( d.time + time_delta_in_millis ) 
// nd := Sat Aug 16 00:00:00 IST 1947 
\end{lstlisting}

Same can easily be achived by the $plusDays()$ function :

\begin{lstlisting}[style=JexlStyle]
d = time('19470815') // Indian Day of Independence  
nd = nd.plusDays(1) // 1947-08-16T00:00:00.000+05:30 
\end{lstlisting}

And times can be subtracted :
\index{chrono : subtraction}

\begin{lstlisting}[style=JexlStyle]
d = time('19470815') // Indian Day of Independence  
nd = d.plusDays(1) // 1947-08-16T00:00:00.000+05:30 
diff = nd - d 
//diff := 86400000 ## Long (millisec gap between two dates)
\end{lstlisting}

So we can see it generates milliseccond gap between two chrono instances.

\end{subsection}

\end{section}

\begin{section}{String : Using str}
\index{str()}
Everything is just a string. It is. Thus every object should be able to be converted to and from out of strings.
Converting an object to a string representation is called \href{https://en.wikipedia.org/wiki/Serialization}{Serialization},
and converting a string back to the object format is called \emph{DeSerialization}.
This is generally done in nJexl by the function $str()$.

\begin{subsection}{Null}
$str()$ never returns null, by design. Thus:
\begin{lstlisting}[style=JexlStyle]
s = str(null)
s == 'null' // true  
\end{lstlisting}
\end{subsection}


\begin{subsection}{Integer}
\index{str : int,INT}
For general integers family, $str()$ acts normally. However, it takes overload
in case of $INT()$ or $BigInteger$ :
\begin{lstlisting}[style=JexlStyle]
bi = INT(42)
s = str(bi) // '42'
s = str(bi,2) // base 2 representation : 101010
\end{lstlisting}
\end{subsection}

\begin{subsection}{Floating Point}
\index{str : float,double,DEC}
For general floating point family, $str()$ acts normally. 
However, it takes overload, which is defined as :

\begin{lstlisting}[style=JexlStyle]
precise_string = str( float_value, num_of_digits_after_decimal )
\end{lstlisting}

To illustrate the point:

\begin{lstlisting}[style=JexlStyle]
d = 101.091891011
str( d, 0 ) // 101 
str(d,1) // 101.1  
str(d,2) // 101.09 
str(d,3) // 101.092 
\end{lstlisting}

\end{subsection}

\begin{subsection}{Chrono}
\index{str : date,time,instant }
Given a chrono family instance, $str()$ can convert them to a format of your choice.
These formats have been already discussed earlier, here they are again:
\begin{enumerate}
\item{Date : \href{https://docs.oracle.com/javase/8/docs/api/java/text/SimpleDateFormat.html}{SimpleDateFormat} }
\item{Time : \href{http://www.joda.org/joda-time/apidocs/org/joda/time/format/DateTimeFormat.html}{DateTimeFormat} }
\end{enumerate}

The syntax is :

\begin{lstlisting}[style=JexlStyle]
formatted_chrono = str( chrono_value , chrono_format )
\end{lstlisting}

Now, some examples :

\begin{lstlisting}[style=JexlStyle]
d = date()
t = time()
i = instant()
str( d ) // default is 'yyyyMMdd' : 20160218
str( t , 'dd - MM - yyyy' ) // 18 - 02 - 2016
str( i , 'dd-MMM-yyyy' ) // 18-Feb-2016 
\end{lstlisting}

\end{subsection}

\begin{subsection}{Collections}
\index{ str : collections }
Collections are formatted by str by default using `,'.
The idea is same for all of the collections, thus we would simply showcase some:

\begin{lstlisting}[style=JexlStyle]
l = [1,2,3,4]
s = str(l) // '1,2,3,4'
s == '1,2,3,4' // true 
s = str(l,'#') // '1#2#3#4'
s == '1#2#3#4' // true 
\end{lstlisting}

So, in essence, for $str()$, serialising the collection, the syntax is :

\begin{lstlisting}[style=JexlStyle]
s = str( collection [ ,  seperation_string ] )
\end{lstlisting}

\end{subsection}

\begin{subsection}{Generalised \emph{toString()} }
\index{ str : replacing toString() }

This brings to the point that we can linearise a collection of collections using $str()$.
Observe how:

\begin{lstlisting}[style=JexlStyle]
l = [1,2,3,4]
m = [ 'a' , 'b' , 'c' ] 
j = l * m // now this is a list of lists, really 
s_form = str{  str($,'#')  }(j, '&') // lineraize 
/* This generates 
1#a&1#b&1#c&2#a&2#b&2#c&3#a&3#b&3#c&4#a&4#b&4#c
*/
\end{lstlisting}

Another way to handle the linearisation is that of dictionary like property buckets:
\index{ str : linearise objects }

\begin{lstlisting}[style=JexlStyle]
d = { 'a' : 10 , 'b' : 20  } 
s_form = str{  [ $.a , $.b ]  }(d,'@' ) 
// lineraize, generates : 10@20
\end{lstlisting}
\end{subsection}

\begin{subsection}{JSON}
\index{ str() : json }\index{json : str() }
Given we have a complex object comprise of primitive types and collection types, 
the \emph{str()} function returns a JSON representation of the complex object.
This way, it is very easy to inter-operate with JavaScript type of languages.

\begin{center}\begin{minipage}{\linewidth}
\begin{lstlisting}[style=JexlStyle]
d = { 'a' : 10 , 'b' : 20  } 
s = str(d)
/*  { "a" : 10 , "b" : 20 }  */
\end{lstlisting}
\end{minipage}\end{center}

\end{subsection}

\end{section}

\begin{section}{Playing with Types : Reflection}
It is essential for a dynamic language to dynamically 
inspect types, if at all. Doing so, is termed as 
\href{https://en.wikipedia.org/wiki/Reflection\_(computer\_programming)}{reflection}.
In this section, we would discuss different constructs that lets one move along the types.


\begin{subsection}{type() function} 
\index{type()}
The function $type()$ gets the type of an object.
In essence it gets the $getClass()$ of any 
\href{https://en.wikipedia.org/wiki/Plain\_Old\_Java\_Object}{POJO}.
Notice that the class object is loaded only once under a class loader,
so equality can be done by simply ``==''.
Thus, the following constructs are of importance:

\begin{center}\begin{minipage}{\linewidth}
\begin{lstlisting}[style=JexlStyle]
d = { : }
c1 = type(d) // c1 := java.util.HashMap
i = 42
c2 = type(i) // c2 := java.lang.Integer
c3 = type(20) // c3 := java.lang.Integer
c3 == c2 // true 
c1 == c2 // false 
\end{lstlisting}
\end{minipage}\end{center}

\end{subsection}

\begin{subsection}{The \emph{===} Operator}
\index{operator : === }
From the earlier section, suppose someone wants to 
equal something with another thing, under the condition
that both the objects are of the same type.
Under the tenet of nJexl which reads : \emph{ ``find a type at runtime, kill the type'' },
we are pretty open into the type business :

\begin{lstlisting}[style=JexlStyle]
a = [1,2,3] 
ca = type(a) // Integer Array 
b = list(1,3,2)
cb = type(b) // XList type 
a == b // true 
ca == cb // false 
// type equals would be :
ca == cb  and a == b // false  
\end{lstlisting}

But that is a long haul. One should be succinct,
so there is this \href{http://www.w3schools.com/js/js\_operators.asp}{borrowed operator from JavaScript},
known as ``==='' which let's its job in a single go :

\begin{lstlisting}[style=JexlStyle]
a = [1,2,3] 
b = list(1,3,2)
a === b // false 
c = [3,1,2]
c === a // true 
\end{lstlisting}
\end{subsection}

\begin{subsection}{The \emph{isa} operator}
\index{operator : isa }
Arguably, finding a type and matching a type is a tough job.
Given that we need to care about who got derived from whatsoever.
That is an issue a language should take care of, and for that reason, 
we do have $isa$ operator.

\begin{lstlisting}[style=JexlStyle]
null isa null // true 
a = [1,2,3] 
a isa [] // false : a is a type of object array?
a isa [0] // true : a is a type of Integer array?
b = list(1,3,2)
b isa list() // true : b is a type of list ? 
\end{lstlisting}
This also showcase the issues of completely overhauling the type 
structure found in a JVM. The first two lines are significantly quirky.

There is another distinctly better way of handling $isa$, by using $alias$ strings,
strings that starts with $@$ and has a type information, e.g. $@map$ :
\index{isa : arr}\index{isa : array} 
\index{isa : map} \index{isa : list}\index{isa : set}\index{isa : Z}\index{isa : Q}\index{isa : num}
\index{isa : chrono}\index{isa : error}
\begin{lstlisting}[style=JexlStyle]
a = [1,2,3] 
a isa "@arr" // true : a is a type that is array?
a isa "@array" // same, true, arr is same as array
b = list(1,3,2)
b isa "@list" // true : b is a type of list ?
{:} isa "@dict" // true 
{:} isa "@map" // true 
0 isa "@Z" // Z : the natural numbers 
0 isa "@num" // num : the numbers 
4.2 isa "@Q" // Q : the rational numbers
s = set(1,2,3)
s isa "@set" // true  
t = time() 
t is a "@chrono" // true : for date/time/instant
e = try{ 0/0 }()
e isa '@error' // true 
\end{lstlisting}
The strings used are case insensitive.
One can pass arbitrary class full name matcher too :
\index{isa : regex matcher} 
\begin{lstlisting}[style=JexlStyle]
'abc' isa '@String' // true 
'abc' isa '@java\.lang\.String' // true 
'abc' isa '@str' // true 
'abc' isa '@foo' // false 
'abc' isa '@s' // false, must take 3 or more letters
\end{lstlisting}
This regex also is case insensitive, and matches from anywhere, 
so be careful. 
\end{subsection}



\begin{subsection}{inspect() function}
\index{inspect()}
A proper reflection requires to \emph{inspect} a particular object type.
The $inspect()$ function comes to foray. The objective of this function 
is to return an \href{https://docs.oracle.com/javase/8/docs/api/java/util/Collections.html#unmodifiableMap-java.util.Map-}{UnmodifiableMap} 
representing the structure of the object. In short, the map returned has these key ingredients :

\begin{enumerate}
\item{The type property : designated as ``$t$''.  Holds the type of the object or class. }
\item{The static fields property : designated as ``$F$''.  
        Holds the static fields of the object or class, with their type. It is a list.}
\item{The instance fields property : designated as ``$f$''.  
        Holds the instance fields of the object or class, with their type. It is a list.}
\item{The instance methods property : designated as ``$m$''.  
        Holds the methods of the object or class, with their name. It is a list.}
\end{enumerate}

This suitably demonstrates the usage:
\begin{center}\begin{minipage}{\linewidth}
\begin{lstlisting}[style=JexlStyle]
d = {'a' : 42 }
id = inspect(d)
/*
{t=java.util.HashMap, 
F=[(UNTREEIFY_THRESHOLD,int), (TREEIFY_THRESHOLD,int), 
(DEFAULT_LOAD_FACTOR,float), 
(DEFAULT_INITIAL_CAPACITY,int), (serialVersionUID,long), 
(MAXIMUM_CAPACITY,int), (MIN_TREEIFY_CAPACITY,int)], 

f=[(entrySet,interface java.util.Set), (threshold,int), 
(modCount,int), (size,int), (loadFactor,float), 
(table,class [Ljava.util.HashMap$Node;)], 

m=[getClass, getOrDefault, newTreeNode, replace, putMapEntries, put, 
containsValue, compute, merge, entrySet, writeObject, containsKey, 
access$000, eq, comparableClassFor, readObject, afterNodeAccess, 
size, loadFactor, newNode, replacementTreeNode, hash, reinitialize, 
internalWriteEntries, wait, values, computeIfAbsent, notifyAll,
 registerNatives, replaceAll, remove, notify, capacity, replacementNode,
  hashCode, get, putAll, putVal, keySet, removeNode, forEach, treeifyBin, 
  clear, isEmpty, tableSizeFor, afterNodeRemoval, computeIfPresent, 
  compareComparables, equals, clone, resize, toString, 
  finalize, getNode, putIfAbsent, afterNodeInsertion]}

*/
\end{lstlisting}
\end{minipage}\end{center}

\end{subsection}

\begin{subsection}{Serialisation using dict() function}
\index{ dict() : serialisation }
Sometimes it is of importance to inspect an instance as 
what the instance is supposed to be, as a property bucket.
That is where the $dict()$ function comes in.

\begin{center}\begin{minipage}{\linewidth}
\begin{lstlisting}[style=JexlStyle]
d = dict()
pd = dict(d,null,null)
/* 
pd := {modCount=1, @t=java.util.HashMap, size=1, 
 loadFactor=0.75, entrySet=[a=42], threshold=12, 
 table=[Ljava.util.HashMap$Node;@7e32c033} 
*/
\end{lstlisting}
\end{minipage}\end{center}

Note the quirky way to inspect what is there for the object when the object is dictionary.
For objects which are not a dictionary, inspecting them is easier :

\begin{center}\begin{minipage}{\linewidth}
\begin{lstlisting}[style=JexlStyle]
d = date()
pd = dict(d)
/* 
pd :=  { cdate=2016-02-29T20:10:06.627+0530, 
         @t=java.util.Date, 
         fastTime=1456756806627 } 
*/
\end{lstlisting}
\end{minipage}\end{center}

\end{subsection}

\begin{subsection}{Field Access}
\index{reflection : field access}
Once we have found the fields to access upon, 
there should be some way to get to the field, other than this :

\begin{lstlisting}[style=JexlStyle]
d = date()
d.fastTime  // some value 
\end{lstlisting}

There is obvious way to access properties
as if the object is a dictionary, or rather than a property bucket:
\index{reflection : property bucket}
\begin{lstlisting}[style=JexlStyle]
d = date()
d['fastTime'] == d.fastTime  // true 
\end{lstlisting}
This opens up countless possibilities of all what one can do with this.
Observe however, monsters like spring and hibernate is not required 
given the property injection is this simple. 
\end{subsection}



\end{section}























