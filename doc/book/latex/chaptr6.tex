\chapter{Types and Conversions}\label{type-conversions}

{\LARGE T}ypes are not much useful for general purpose programming, save that they avoid errors.
Sometimes they are necessary, and some types are indeed useful because they let us do 
many useful stuff. In this chapter we shall talk about these types and how to convert
one to another.

The general form for type casting can be written :
\begin{lstlisting}[style=JexlStyle][float]
val = type_function(value, optional_default_value = null )
\end{lstlisting}

How does this work? The system would try to cast $value$
into the specific type. If it failed, and there is no default value, 
it would return $null$. However, if default is passed, it would return 
that when conversion fails.
This neat design saves a tonnage of $try \; ... \; catch$ stuff. 

\begin{section}{Integer Family}
This section is dedicated to the natural numbers family.
We have :
\begin{enumerate}
\item{bool}
\item{short}
\item{char}
\item{int}
\item{long}
\item{INT : Java BigInteger}
\end{enumerate}

\begin{subsection}{Boolean}\index{bool()}
The syntax is :
\begin{lstlisting}[style=JexlStyle][float]
val = bool(value, optional_default_value = null )
val = bool(value, optional_matching_values[2])
\end{lstlisting}
Observe both in action :

\begin{lstlisting}[style=JexlStyle][float]
val = bool("hello") // val is null
val = bool("hello",false) // val is false
val = bool('hi', ['hi' , 'bye' ] ) // val is true 
val = bool('bye', ['hi' , 'bye' ] ) // val is false 
\end{lstlisting}

\end{subsection}

\begin{subsection}{Short}\index{short()}
This is almost never required, and is there for backward compatibility with Java data types.
The syntax is :

\begin{lstlisting}[style=JexlStyle][float]
val = short(value, optional_default_value = null )
\end{lstlisting}

Usage is :

\begin{lstlisting}[style=JexlStyle][float]
val = short("hello") // val is null
val = short("hello",0) // val is 0
val = short('42') // val is 42 
val = short(42) // val is 42 
\end{lstlisting}

\end{subsection}

\begin{subsection}{Character}\index{char()}
For almost all practical purposes, character is noting but the short value, 
interpreted by a code page.

The syntax is :

\begin{lstlisting}[style=JexlStyle][float]
val = char(value, optional_default_value = null )
\end{lstlisting}

Usage is :

\begin{lstlisting}[style=JexlStyle][float]
val = char("hello") // val is "h"
val = char(121231231,0) // val is 0
val = char('4') // val is '4' 
// ascii stuff?
val = char(65) // val is 'A'  
\end{lstlisting}

Generally none needs to get into this, because generally $String.charAt(index)$ is a good substitute
for finding a character.
\end{subsection}

\begin{subsection}{Integer}\index{int()}
This is very useful and the syntax is :

\begin{lstlisting}[style=JexlStyle][float]
val = int(value, optional_default_value = null )
\end{lstlisting}

Usage is :

\begin{lstlisting}[style=JexlStyle][float]
val = int("hello") // val is null
val = int("hello",0) // val is 0
val = int('42') // val is 42 
val = int(42) // val is 42 
val = int ( 10.1 ) // val is 10 
val = int ( 10.9 ) // val is 10 
\end{lstlisting}

\end{subsection}


\begin{subsection}{Long}\index{long()}
This is rarely required, and the syntax is :

\begin{lstlisting}[style=JexlStyle][float]
val = long(value, optional_default_value = null )
\end{lstlisting}

Usage is :

\begin{lstlisting}[style=JexlStyle][float]
val = long("hello") // val is null
val = long("hello",0) // val is 0
val = long('42') // val is 42 
val = long(42) // val is 42 
val = long( 10.1 ) // val is 10 
val = long( 10.9 ) // val is 10 
\end{lstlisting}

\end{subsection}

\begin{subsection}{BigInteger}\index{INT()}
This is sometimes required, and the syntax is :

\begin{lstlisting}[style=JexlStyle][float]
val = INT(value, base=10, default_value = null )
\end{lstlisting}

Usage is :

\begin{lstlisting}[style=JexlStyle][float]
val = INT("hello") // val is null
val = INT('hi',10,42 )// val is 42
val = INT('42') // val is 42 
val = INT(54,13 ) // val is 42 
\end{lstlisting}

\end{subsection}
\end{section}


\begin{section}{Rational Numbers Family}
This section is dedicated to the floating point numbers family.
We have :
\begin{enumerate}
\item{float}
\item{double}
\item{BigDecimal}
\end{enumerate}

\begin{subsection}{Float}\index{float()}
This is not very useful and the syntax is :

\begin{lstlisting}[style=JexlStyle][float]
val = float(value, optional_default_value = null )
\end{lstlisting}

Usage is :

\begin{lstlisting}[style=JexlStyle][float]
val = float("hello") // val is null
val = float("hello",0) // val is 0.0
val = float('42') // val is 42.0 
val = float(42) // val is 42.0 
val = float ( 10.1 ) // val is 10.1 
val = float ( 10.9 ) // val is 10.9 
\end{lstlisting}

\end{subsection}


\begin{subsection}{Double}\index{double()}
This is generally required, and the syntax is :

\begin{lstlisting}[style=JexlStyle][float]
val = double(value, optional_default_value = null )
\end{lstlisting}

Usage is :

\begin{lstlisting}[style=JexlStyle][float]
val = double("hello") // val is null
val = double("hello",0) // val is 0.0
val = double('42') // val is 42.0 
val = double(42) // val is 42.0 
val = double( 10.1 ) // val is 10.1 
val = double( 10.9 ) // val is 10.9 
\end{lstlisting}

\end{subsection}

\begin{subsection}{BigDecimal}\index{DEC()}
This is sometimes required, and the syntax is :

\begin{lstlisting}[style=JexlStyle][float]
val = DEC(value,default_value = null )
\end{lstlisting}

Usage is :

\begin{lstlisting}[style=JexlStyle][float]
val = DEC("hello") // val is null
val = DEC('hi',10,42 )// val is 42.0
val = DEC('42') // val is 42.0 
val = DEC(42.00001 ) // val is 42.00001
\end{lstlisting}

\end{subsection}

\end{section}

\begin{section}{The Chrono Family}
\index{date}\index{time}\index{instant}\index{format : date and time}

Handling date and time has been a problem, that too with timezones.
nJexl simplifies the stuff. 
We have three basic types to handle date/time:
\begin{enumerate}
\item{date : $java.lang.Date$ : because of Java compatibility with SDKs. }
\item{time : $org.joda.time.DateTime$ : that is the best one out there. }
\item{instant : $java.time.Instant$ : for newer systems who wants to experiment. }
\end{enumerate}


\begin{subsection}{Date}\index{date()}
This is how you create a date:
\begin{lstlisting}[style=JexlStyle][float]
val = date([ value, date_format ] )
\end{lstlisting}

With no arguments, it gives the current date time:

\begin{lstlisting}[style=JexlStyle][float]
today = date()
\end{lstlisting}

The default date format is $yyyyMMdd$, so :
\begin{lstlisting}[style=JexlStyle][float]
dt = date('20160218') // Thu Feb 18 00:00:00 IST 2016
\end{lstlisting}

For all the date formats on dates which are supported, 
see \href{https://docs.oracle.com/javase/8/docs/api/java/text/SimpleDateFormat.html}{SimpleDateFormat}.

Take for example :

\begin{lstlisting}[style=JexlStyle][float]
dt = date('2016/02/18', 'yyyy/MM/dd' ) // Thu Feb 18 00:00:00 IST 2016
dt = date('2016-02-18', 'yyyy-MM-dd' ) // Thu Feb 18 00:00:00 IST 2016
\end{lstlisting}


\end{subsection}

\begin{subsection}{Time}\index{time()}

This is how you create a joda \href{http://joda-time.sourceforge.net/apidocs/org/joda/time/DateTime.html}{DateTime}:
\begin{lstlisting}[style=JexlStyle][float]
val = time([ value, date_format , time_zone] )
\end{lstlisting}

With no arguments, it gives the current date time:

\begin{lstlisting}[style=JexlStyle][float]
today = time()
\end{lstlisting}

The default date format is $yyyyMMdd$, so :
\begin{lstlisting}[style=JexlStyle][float]
dt = time('20160218') // 2016-02-18T00:00:00.000+05:30
\end{lstlisting}

For all the date formats on dates which are supported, 
see \href{http://www.joda.org/joda-time/apidocs/org/joda/time/format/DateTimeFormat.html}{DateTimeFormat}.

Take for example :

\begin{lstlisting}[style=JexlStyle][float]
dt = time('2016/02/18', 'yyyy/MM/dd' ) // 2016-02-18T00:00:00.000+05:30
dt = time('2016-02-18', 'yyyy-MM-dd' ) // 2016-02-18T00:00:00.000+05:30
\end{lstlisting}

If you want to convert one time to another timezone, you need to give the time, 
and the \href{http://joda-time.sourceforge.net/timezones.html}{timezone}:

\begin{lstlisting}[style=JexlStyle][float]
dt = time()
dt_honolulu = time(dt , 'Pacific/Honolulu' ) // 2016-02-17T17:23:02.754-10:00 
dt_ny = time( dt, 'America/New_York' ) // 2016-02-17T22:23:02.754-05:00
\end{lstlisting}

\end{subsection}


\begin{subsection}{Instant}\index{instant()}

With no arguments, it gives the current instant time:

\begin{lstlisting}[style=JexlStyle][float]
today = instant()
\end{lstlisting}

It is freely mixable with other chrono types.

\end{subsection}


\begin{subsection}{Comparison on Chronos}\index{operator : compare, chrono }
All these date time types are freely mixable, 
and all comparison operations are defined with them.
Thus :

\begin{lstlisting}[style=JexlStyle][float]
d = date() // wait for some time 
t = time() // wait for some time 
i = instant()
// now compare 
c = ( d < t and t < i ) // true 
c = ( i > t and t > d ) // true 
\end{lstlisting}

Tow dates can be equal to one another, but not two instances,
that is a very low probability event, almost never.
Thus, equality makes sense when we know it is date, and not instant :

\begin{lstlisting}[style=JexlStyle][float]
d = date('19470815') // Indian Day of Independence  
t = time('19470815') // Indian Day of Independence  
// now compare 
c = ( d == t ) // true 
\end{lstlisting}

\end{subsection}

\begin{subsection}{Arithmetic on Chronos}
\index{arithmetic : chorono}
Dates, obviously can not be multiplied or divided.
That would be a sin. However, dates can be added with other reasoanable values, 
dates can be subtracted from one another, and $time()$ can be added or subtracted by days, months or even year.
More of what all date time supports, 
see the manual of \href{http://joda-time.sourceforge.net/apidocs/org/joda/time/DateTime.html}{DateTime}.

To add time to a date, there is another nifty method :

\begin{lstlisting}[style=JexlStyle][float]
d = date('19470815') // Indian Day of Independence  
time_delta_in_millis = 24 * 60 * 60 * 1000 // next day 
nd = date( d.time + time_delta_in_millis ) // Sat Aug 16 00:00:00 IST 1947 
\end{lstlisting}

Same can easily be achived by the $plusDays()$ function :

\begin{lstlisting}[style=JexlStyle][float]
d = time('19470815') // Indian Day of Independence  
nd = nd.plusDays(1) // 1947-08-16T00:00:00.000+05:30 
\end{lstlisting}
\end{subsection}

\end{section}

\begin{section}{String : Using str}
\index{str()}
Everything is just a string. It is. Thus every object should be able to be converted to and from out of strings.
Converting an object to a string representation is called \href{https://en.wikipedia.org/wiki/Serialization}{Serialization},
and converting a string back to the object format is called \emph{DeSerialization}.
This is generally done in nJexl by the function $str()$.

\begin{subsection}{Null}
$str()$ never returns null, by design. Thus:
\begin{lstlisting}[style=JexlStyle][float]
s = str(null)
s == 'null' // true  
\end{lstlisting}
\end{subsection}


\begin{subsection}{Integer}
\index{str : int,INT}
For general integers family, $str()$ acts normally. However, it takes overload
in case of $INT()$ or $BigInteger$ :
\begin{lstlisting}[style=JexlStyle][float]
bi = INT(42)
s = str(bi) // '42'
s = str(bi,2) // base 2 representation : 101010
\end{lstlisting}
\end{subsection}

\begin{subsection}{Floating Point}
\index{str : float,double,DEC}
For general floating point family, $str()$ acts normally. 
However, it takes overload, which is defined as :

\begin{lstlisting}[style=JexlStyle][float]
precise_string = str( float_value, num_of_digits_after_decimal )
\end{lstlisting}

To illustrate the point:

\begin{lstlisting}[style=JexlStyle][float]
d = 101.091891011
str( d, 0 ) // 101 
str(d,1) // 101.1  
str(d,2) // 101.09 
str(d,3) // 101.092 
\end{lstlisting}

\end{subsection}

\begin{subsection}{Chrono}
\index{str : date,time,instant }
Given a chrono family instance, $str()$ can convert them to a format of your choice.
These formats have been already discussed earlier, here they are again:
\begin{enumerate}
\item{Date : \href{https://docs.oracle.com/javase/8/docs/api/java/text/SimpleDateFormat.html}{SimpleDateFormat} }
\item{Time : \href{http://www.joda.org/joda-time/apidocs/org/joda/time/format/DateTimeFormat.html}{DateTimeFormat} }
\end{enumerate}

The syntax is :

\begin{lstlisting}[style=JexlStyle][float]
formatted_chrono = str( chrono_value , chrono_format )
\end{lstlisting}

Now, some examples :

\begin{lstlisting}[style=JexlStyle][float]
d = date()
t = time()
i = instant()
str( d ) // default is 'yyyyMMdd' : 20160218
str( t , 'dd - MM - yyyy' ) // 18 - 02 - 2016
str( i , 'dd-MMM-yyyy' ) // 18-Feb-2016 
\end{lstlisting}

\end{subsection}

\begin{subsection}{Collections}
\index{ str : collections }
Collections are formatted by str by default using `,'.
The idea is same for all of the collections, thus we would simply showcase some:

 


\end{subsection}

\begin{subsection}{Generalized $toString()$ }
\end{subsection}


\end{section}

























