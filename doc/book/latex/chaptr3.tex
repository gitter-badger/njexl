\chapter{nJexl in 3 Minutes}\label{intro-njexl}

{\LARGE W}ith some understanding on C,C++, Java, then it will be very easy for you to learn nJexl. The biggest syntactic difference between nJexl and other languages is that the ';' statement end character is optional. When we consider a nJexl program it can be defined as a collection of objects that communicate via invoking each others methods. 


\begin{section}{Building Blocks}

\begin{subsection}{Identifiers}
nJexl is case-sensitive, which means identifier Hello and hello would have different meaning.
All nJexl components require names. Names used for objects, classes, variables and methods are called identifiers. 
A keyword cannot be used as an identifier and identifiers are case-sensitive. 
See here for a list of keywords. 

\end{subsection}


\begin{subsection}{Assignments}
Most basic syntax of nJexl is, like any other language : assignment.
\begin{lstlisting}[style=JexlStyle][float]
 a = 1 //  assigns local variable a to Integer 1
 b = 1.0 //assigns local variable a to b float 1.0
 c = 'Hello, nJexl' // assigns local variable c to String 'Hello, nJexl'
 d = "Hello, nJexl" ## same, strings are either single or double quoted 
 /* 
 assigns the *then* value of a to e, 
 subsequent change in a wont reflect in e 
 */
 e = a \end{lstlisting}

\end{subsection}

\begin{subsection}{Comments}
See from the previous subsection "//" used as line comments. So is "\#\#".
Along with the multiline comment "/*" with  "*/" :

\end{subsection}


\begin{subsection}{Basic Types}

Basic types are :
\begin{lstlisting}[style=JexlStyle][float]
 a = 1 // Integer 
 b = 1.0 //  float 
 c = 'Hello, nJexl' // String 
 d = 1.0d ## Double 
 I = 1h // BigInteger 
 D = 1.0D // BigDecimal  
 tt = true // boolean 
 tf = false // boolean 
\end{lstlisting}

\end{subsection}

\begin{subsection}{Multiple Assignment}
nJexl supports multiple assignment. It has various usage:

\begin{lstlisting}[style=JexlStyle][float]
 a = 1 // Integer 
 b = 1.0 //  float 
 c = 'Hello, nJexl' // String 
 // instead, do this straight :
 #(a,b,c ) = [ 1 , 1.0 , 'Hello, nJexl' ]   
\end{lstlisting}
\end{subsection}
\end{section}

\begin{section}{operators}

\begin{subsection}{Arithmetic}

\begin{lstlisting}[style=JexlStyle][float]
a = 1 + 1 // addition : a <- 2
z = 1 - 1 // subtraction : z <- 0
m = 2 * 3 // multiply : m <- 6
d = 3.0 / 2.0 // divide d <- 1.5   
x = 2 ** 10 // Exponentiation x <- 1024
y = -x // negation, y <- -1024 
r = 3 % 2 // modulo, r <- 1 
r = 3 mod 2 // modulo, r <- 1    
\end{lstlisting}

\end{subsection}

\begin{subsection}{Logical}

\begin{lstlisting}[style=JexlStyle][float]
o = true or true // true , or operator 
o = true || true // same 
a = true and false // false , and operator 
a = true && false // false, and operator  
\end{lstlisting}

\end{subsection}


\begin{subsection}{Comparison}

\begin{lstlisting}[style=JexlStyle][float]
t = 10 < 20 // true, less than
t = 10 lt < 20 // true , less than 
f = 10 > 20 // false, greater then 
f = 10 gt 20 // false, greater then 
t = 10 <= 10 // true, less than or equal to 
t = 10 le < 10 // true , less than or equal to 
t = 10 >= 10 // true, greater then or equal to 
t = 10 ge 10 // true, greater then or equal to 
t = ( 10 == 10 ) // true, equal to 
t = ( 10 eq 10 ) // true, equal to 
f = ( 10 != 10 ) // false, not equal to 
f = ( 10 ne 10 ) // false, not equal to 
\end{lstlisting}

\end{subsection}

\end{section}


\begin{section}{Conditions}

People coming from any other language would find them trivial.

\begin{subsection}{IF}

\begin{lstlisting}[style=JexlStyle][float]
x = 10 
if ( x < 100 ){
   x = x**2
}
write(x) // writes back x to standard output : 100
\end{lstlisting}
\end{subsection}

\begin{subsection}{ELSE}
\begin{lstlisting}[style=JexlStyle][float]
x = 1000 
if ( x < 100 ){
   x = x**2
}else{
  x = x/10 
}
write(x) // writes back x to standard output : 100
\end{lstlisting}
\end{subsection}

\begin{subsection}{ELSE IF}
\begin{lstlisting}[style=JexlStyle][float]
x = 100 
if ( x < 10 ){
   x = x**2
} else if( x > 80  ){
  x = x/10 
} else {
   x = x/100 
}
write(x) // writes back x to standard output : 10
\end{lstlisting}
\end{subsection}

\begin{subsection}{GOTO}
\end{subsection}

\end{section}

\begin{section}{Loops}

\begin{subsection}{WHILE}
\end{subsection}

\begin{subsection}{FOR}
\end{subsection}

\end{section}

\begin{section}{Functions}

\begin{subsection}{Defining}
\end{subsection}

\begin{subsection}{Calling}
\end{subsection}

\begin{subsection}{Global Variables}
\end{subsection}

\end{section}

\begin{section}{Anonymous Function as Parameter}

\begin{subsection}{Why it is needed?}
\end{subsection}

\begin{subsection}{Some Use Cases}
\end{subsection}

\end{section}

\begin{section}{Available Data Structures}

\begin{subsection}{Array}
\end{subsection}

\begin{subsection}{List}
\end{subsection}

\begin{subsection}{Set}
\end{subsection}

\begin{subsection}{Dict}
\end{subsection}

\end{section}

